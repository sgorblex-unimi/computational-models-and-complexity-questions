\chapter{Computational models}
The language accepted by a machine $M$ is denoted by $\genlang(M)$.

% A machine working over a single letter alphabet is called unary, as its generated language.

\begin{defn}
	A machine with transition function $\delta:S\to\subsets{T}$ is said to be deterministic whenever $\card{\delta(s)}\le1$, for any $s\in S$.

	In this case, $\delta$ can be interpreted as a partial function, with the notation $\delta(s)=t$ used as a shorthand for $\delta(s)=\set{t}$.
\end{defn}

\begin{defn}[one-way finite automata]
	A \emph{(one-way) nondeterministic finite automaton} (1NFA or simply NFA) is a quintuple $(Q,\Sigma,\delta,q_I,F)$ where:
	\begin{itemize}
		\item $Q$ is the finite and nonempty set of \emph{states},
		\item $\Sigma$ is the finite \emph{input alphabet},
		\item $\delta:Q\times\Sigma\to\subsets Q$ is the \emph{transition function},
		\item $q_I\in Q$ is the \emph{initial state}, and
		\item $F\subseteq Q$ is the set of \emph{accepting} (or \emph{final}) \emph{states}.
	\end{itemize}
	The input word $w$ is read from left to right, one symbol at a time.
	With each step, the machine changes its state, with the transition function $\delta$ representing the possible next states based on the current state and the symbol read.
	The NFA accepts an input string $w=w_1\cdots w_n$ if there is a sequence of states $q_0,q_1,\dots,q_n$ such that $q_0=q_I$, $q_n\in F$, and for each $i\in\set{0,1,\dots,n-1}$, $q_{i+1}\in\delta(q_i,w_i)$.

	The deterministic version of 1NFA is indicated with 1DFA, or simply DFA.
\end{defn}



\begin{defn}[two-way finite automata]
	A \emph{two-way nondeterministic finite automaton} (2NFA) is a quintuple $(Q,\Sigma,\delta,q_I,F)$ where:
	\begin{itemize}
		\item $Q$, $\Sigma$, $q_I$ and $F$ are defined as for 1NFAs, and
		\item $\delta:Q\times(\Sigma\cup\set{\lem,\rem})\to\subsets{Q\times\set{\tl,\tr}}$ is the \emph{transition function}, with the two special symbols $\lem,\rem\notin\Sigma$ called the left and the right end-markers, respectively,
	\end{itemize}
	The input word $w$ is stored onto a tape surrounded by the two end-markers, the left end-marker being at position zero and the right end-marker at position $\len w+1$.
	The tape has a head initially pointing at $\lem$.
	In one move, the machine reads an input symbol, changes its state, and moves the tape head one position left or right, depending on whether $\delta$ returns $\tl$ or $\tr$, respectively.
	Furthermore, the head cannot surpass the end-markers, except at the end of computation, to accept the input.
	The machine accepts the input if there exists a computation path that starts in the initial state $q_I$ and ends in a final state $q_f\in F$ after moving right from $\rem$.

	The deterministic version of 2NFA is indicated with 2DFA.
\end{defn}





\begin{defn}[limited automata]\label{def:kla}
	Given an integer $k\ge0$, a \emph{$k$-limited automaton} ($k$-LA) is a 6-tuple $(Q,\Sigma,\Gamma,\delta,q_I,F)$ where:
	\begin{itemize}
		\item $Q$, $\Sigma$, $q_I$ and $F$ are defined as for 2NFAs,
		\item $\Gamma$ is the finite \emph{working alphabet}, partitioned into $\Gamma_0\cup\Gamma_1\cup\cdots\cup\Gamma_k$, where $\Gamma_0=\Sigma$, and
		\item $\delta:Q\times\Gamma\to\subsets{Q\times(\Gamma\setminus\set{\lem,\rem})\times\set{\tl,\tr}}$ is the \emph{transition function},
	\end{itemize}
	The tape is initially setup as for 2NFAs.
	In one move the machine, besides changing its state and moving the tape head, changes the content of the cell under the head according to the transition function.

	Furthermore, the transition function is subject to restrictions which, essentially, allow to modify the content of a cell only during the first $k$ visits.
	In order to implement these restrictions, $\delta$ is required to satisfy the following condition. For each $(t,y,d)\in\delta(s,x)$, with $x\in\Gamma_i$:
	\begin{enumerate}[(1)]
		\item if $i=k$ then $x=y$,
		\item if $i<k$ and $d=\tr$ then $y\in\Gamma_j$, with $j=\min\set{\ceil{\frac{i}{2}}\cdot 2+1,k}$, and
		\item if $i<k$ and $d=\tl$ then $y\in\Gamma_j$, with $j=\min\set{\ceil{\frac{i+1}{2}}\cdot 2,k}$.
	\end{enumerate}

	\noindent Acceptance is defined as for 2NFAs.

	\noindent The deterministic version of $k$-LA is indicated with $k$-DLA.
\end{defn}



\begin{defn}\label{def:sweeping}
	A two-way machine is said to be \emph{sweeping} if the direction of the head movement only changes on the end-markers.
	The computation is therefore divided into traversals of the entire tape, also called \emph{sweeps}.
\end{defn}
