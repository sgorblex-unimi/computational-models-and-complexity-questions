\chapter{Complexity results for regular language acceptors}\label{app:results}
In this chapter we summarize previous results about the cost of simulations (conversions) between various regular language acceptors: finite state automata and $1$-limited automata.
We represent such costs in tables, first in \Cref{sec:prevsims-general} for the general case (\Cref{tab:sims-core-general,tab:sims-1la-general}), and second in \Cref{sec:prevsims-unary} for the unary case (\Cref{tab:sims-core-unary,tab:sims-1la-unary}).

Each cell contains upper and lower bounds for the cost of the simulation of a machine of the kind given by the row by a machine of the kind given by the column.
When the bounds match, one cost is displayed (two exponential bounds will be considered to match even if the match is not exact).
The simulations marked $\Ctriv$ are ones where the simulated machine is virtually a subcase of the simulating one, therefore close to no size growth ($O(n)$) is required.



\section{The general case}\label{sec:prevsims-general}

\begin{table}
	\centering
	\renewcommand{\hstdef}{.35}
\begin{tabular}{|l|l|l|p{2.9em}|l|}
	\hline
	~     & \ODFA   & \ONFA     & \TDFA                          & \TNFA     \\ \hline
	\ODFA & \cY     & $\Cctriv$ & $\Cctriv$                      & $\Cctriv$ \\ \hline
	\ONFA & $\Cexp$ & \cY       & \cR $\le\Cexp\hst$ $\ge\Cpoly$ & $\Cctriv$ \\ \hline
	\TDFA & $\Cexp$ & $\Cexp$   & \cY                            & $\Cctriv$ \\ \hline
	\TNFA & $\Cexp$ & $\Cexp$   & \cR $\le\Cexp\hst$ $\ge\Cpoly$ & \cY       \\ \hline
\end{tabular}
%
	\caption{Descriptional complexity of the simulations between basic regular language recognisers, general case.}
	\label{tab:sims-core-general}
\end{table}

\Cref{tab:sims-core-general} contains the costs of simulations between basic regular language acceptors.
The cells highlighted in red represent the unsolved Sakoda and Sipser problem \cite{SakSip78} about the cost of removal of nondeterminism from finite state automata.
Here we give a description of each nontrivial result:

\paragraph{\ONFA{}\tto\ODFA}\label{cost:1NFAto1DFA}
\begin{itemize}
	\item $\le\Cexp$: by using the subset construction presented by \triplecite{RabSco59},
	\item $\ge\Cexp$: distinguishability arguments on the witness languages recognised by Meyer and Fischer's automaton \cite{MeyFis71} and Moore's automaton \cite{Moo71}.
\end{itemize}
\paragraph{\ONFA{}\tto\TDFA}
\begin{itemize}
	\item $\le\Cexp$: by using construction for \hyperref[cost:1NFAto1DFA]{\ONFA{}\tto\ODFA}, which is $\le\Cexp$, combined with $\Ctriv$ \ODFA{}\tto\TDFA,
	\item believed to be $=\Cexp$ (Sakoda and Sipser conjecture \cite{SakSip78}).
\end{itemize}
\paragraph{\TDFA{}\tto\ODFA}\label{cost:2DFAto1DFA}
\begin{itemize}
	\item $\le\Cexp$: by using the transition table construction presented by \triplecite{She59} or the crossing sequence construction presented in parallel by \doublecite{RabSco59},
	\item $\ge\Cexp$: otherwise \hyperref[cost:2DFAto1NFA]{\TDFA{}\tto\ONFA} would be less than $\Cexp$ since \ODFA{}\tto\ONFA is $\Ctriv$ (see lower bound below).
\end{itemize}
\paragraph{\TDFA{}\tto\ONFA}\label{cost:2DFAto1NFA}
\begin{itemize}
	\item $\le\Cexp$: by using construction for \hyperref[cost:2DFAto1DFA]{\TDFA{}\tto\ODFA}, which is $\le\Cexp$, combined with $\Ctriv$ \ODFA{}\tto\ONFA,
	\item $\ge\Cexp$: proven by \triplecite{Bir93}.
\end{itemize}
\paragraph{\TNFA{}\tto\ODFA}\label{cost:2NFAto1DFA}
\begin{itemize}
	\item $\le\Cexp$: generally attributed to the same articles of \hyperref[cost:2DFAto1DFA]{\TDFA{}\tto\ODFA},
	\item $\ge\Cexp$: otherwise \hyperref[cost:1NFAto1DFA]{\ONFA{}\tto\ODFA} would be less than $\Cexp$ since \ONFA{}\tto\TNFA is $\Ctriv$.
\end{itemize}
\paragraph{\TNFA{}\tto\ONFA}
\begin{itemize}
	\item $=\Cexp$: both upper and lower bounds were proven by \triplecite{Kap05}. Precisely, the optimal cost is $\binom{2n}{n+1}$.
\end{itemize}
\paragraph{\TNFA{}\tto\TDFA}
\begin{itemize}
	\item $\le\Cexp$: by using construction for \hyperref[cost:2NFAto1DFA]{\TNFA{}\tto\ODFA}, which is $\le\Cexp$, combined with $\Ctriv$ \ODFA{}\tto\TDFA,
	\item believed to be $=\Cexp$ (Sakoda and Sipser conjecture \cite{SakSip78}).
\end{itemize}


\begin{table}
	\centering
	\renewcommand{\hstdef}{.55}
\begin{tabular}{|l|l|l|p{3.1em}|l|l|p{3.1em}|}
	\hline
	~     & \ODFA              & \ONFA     & \TDFA                                            & \TNFA     & \OLA      & \ODLA                                            \\ \hline
	\ODFA & \cY                & $\Cctriv$ & $\Cctriv$                                        & $\Cctriv$ & $\Cctriv$ & $\Cctriv$                                        \\ \hline
	\ONFA & $\Cexp$            & \cY       & \cR $\le\Cexp\hst$ $\ge\Cpoly$                   & $\Cctriv$ & $\Cctriv$ & \cB $\le\Cexp\hst$ $\ge\Cpoly$                   \\ \hline
	\TDFA & $\Cexp$            & $\Cexp$   & \cY                                              & $\Cctriv$ & $\Cctriv$ & $\Cctriv$                                        \\ \hline
	\TNFA & $\Cexp$            & $\Cexp$   & \cR $\le\Cexp\hst$ $\ge\Cpoly$                   & \cY       & $\Cctriv$ & \cB $\le\Cexp\hst$ $\ge\Cpoly$                   \\ \hline
	\OLA  & \rbt[.2]{$\Cdexp$} & $\Cexp$   & \cG \rbt[.2]{$\le\Cdexp\hst[.1]$} $\ge\Cexp\hst$ & $\Cexp$   & \cY       & \cG \rbt[.2]{$\le\Cdexp\hst[.1]$} $\ge\Cexp\hst$ \\ \hline
	\ODLA & $\Cexp$            & $\Cexp$   & $\Cexp$                                          & $\Cexp$   & $\Cctriv$ & \cY                                              \\ \hline
\end{tabular}
%
	\caption{Descriptional complexity of the simulations between basic regular language recognisers and $1$-limited automata, general case.}
	\label{tab:sims-1la-general}
\end{table}

\Cref{tab:sims-1la-general} contains the costs of simulations between basic regular language acceptors and $1$-limited automata.
In addition to the cells highlighted in red for the Sakoda and Sipser conjecture, we have blue and green cells.
The cells highlighted in blue consist in a relaxed version of the problem, with an extension of \TNFA, namely \ODLA, as the simulating machine.
The cells highlighted in green consist in a harder version of the problem, with an extension of \NFA, namely \OLA, as the simulated machine.
Here we give a description of each nontrivial result that hasn't been explained previously:

\paragraph{\ONFA{}\tto\ODLA}
\begin{itemize}
	\item $\le\Cexp$: by using construction for \hyperref[cost:1NFAto1DFA]{\ONFA{}\tto\ODFA}, which is $\le\Cexp$, combined with $\Ctriv$ \ODFA{}\tto\ODLA,
	\item relaxed form of the Sakoda and Sipser conjecture with extended \TDFA.
\end{itemize}
\paragraph{\TNFA{}\tto\ODLA}
\begin{itemize}
	\item $\le\Cexp$: by using construction for \hyperref[cost:2NFAto1DFA]{\TNFA{}\tto\ODFA}, which is $\le\Cexp$, combined with $\Ctriv$ \ODFA{}\tto\ODLA,
	\item relaxed form of the Sakoda and Sipser conjecture with extended \TDFA.
\end{itemize}
\paragraph{\OLA{}\tto\ODFA}\label{cost:1LAto1DFA}
\begin{itemize}
	\item $\le\Cdexp$: by using the transition table construction presented by \triplecite{PigPis14},
	\item $\ge\Cdexp$: distinguishability arguments on witness languages: the block language $L_n$ defined by \triplecite{PigPis14} and the "reset" witness languages by \triplecite{PigPri+22}.
\end{itemize}
\paragraph{\OLA{}\tto\ONFA}\label{cost:1LAto1NFA}
\begin{itemize}
	\item $\le\Cexp$: by using the transition table construction presented by \triplecite{PigPis14},
	\item $\ge\Cexp$: otherwise \hyperref[cost:1LAto1DFA]{\OLA{}\tto\ODFA} would be less than $\Cdexp$ since \hyperref[cost:1NFAto1DFA]{\ONFA{}\tto\ODFA} is $\Cexp$.
\end{itemize}
\paragraph{\OLA{}\tto\TDFA}
\begin{itemize}
	\item $\le\Cdexp$: by using construction for \hyperref[cost:1LAto1DFA]{\OLA{}\tto\ODFA}, which is $\le\Cdexp$, combined with $\Ctriv$ \ODFA{}\tto\TDFA,
	\item $\ge\Cexp$: otherwise \hyperref[cost:1LAto1DFA]{\OLA{}\tto\ODFA} would be less than $\Cdexp$ since \hyperref[cost:2DFAto1DFA]{\TDFA{}\tto\ODFA} is $\Cexp$,
	\item harder form of the Sakoda and Sipser conjecture with extended \TNFA.
\end{itemize}
\paragraph{\OLA{}\tto\TNFA}
\begin{itemize}
	\item $\le\Cexp$: by using construction for \hyperref[cost:1LAto1NFA]{\OLA{}\tto\ONFA}, which is $\le\Cexp$, combined with $\Ctriv$ \ONFA{}\tto\TNFA,
	\item $\ge\Cexp$: otherwise \hyperref[cost:1LAto1DFA]{\OLA{}\tto\ODFA} would be less than $\Cdexp$ since \TNFA{}\tto\ODFA is $\Cexp$.
\end{itemize}
\paragraph{\ODLA{}\tto\ODFA}\label{cost:1DLAto1DFA}
\begin{itemize}
	\item $\le\Cexp$: by using the transition table construction presented by \triplecite{PigPis14},
	\item $\ge\Cexp$: otherwise with \hyperref[cost:1DLAto1DFA]{\ODLA{}\tto\ODFA} we would obtain a \ODFA for $L_n$ with $<\Cdexp$ states (impossible by distinguishability), as a \ODLA (\TDFA) of size $\Cexp$ accepts $L_n$ \cite{PigPis14}.
\end{itemize}
\paragraph{\OLA{}\tto\ODLA}
\begin{itemize}
	\item $\le\Cdexp$: by using construction for \hyperref[cost:1LAto1DFA]{\OLA{}\tto\ODFA}, which is $\le\Cdexp$, combined with $\Ctriv$ \ODFA{}\tto\ODLA,
	\item $\ge\hspacingfix{-1pt}\Cexp$: otherwise \hyperref[cost:1LAto1DFA]{\OLA{}\tto\ODFA} would be less than $\Cdexp$ since \ODLA{}\tto\ODFA is $\Cexp$,
	\item variation of the Sakoda and Sipser problem with both extended \TNFA and \TDFA.
\end{itemize}
\paragraph{\ODLA{}\tto\TNFA}\label{cost:1DLAto2NFA}
\begin{itemize}
	\item $\le\Cexp$: by using construction for \hyperref[cost:1DLAto1DFA]{\ODLA{}\tto\ODFA}, which is $\le\Cexp$, combined with $\Ctriv$ \ODFA{}\tto\TNFA,
	\item $\ge\Cexp$: since the unary witness language $\set{a^{2^n}}\star$ is recognized by the linear size \ODLA presented by \triplecite{PigPri19}, but cannot be recognised by a \TNFA with less than $\Cexp$ states, as proven by \triplecite{MerPig00}.
\end{itemize}
\paragraph{\ODLA{}\tto\ONFA}
\begin{itemize}
	\item $\le\Cexp$: by using construction for \hyperref[cost:1DLAto1DFA]{\ODLA{}\tto\ODFA}, which is $\le\Cexp$, combined with $\Ctriv$ \ODFA{}\tto\ONFA,
	\item $\ge\Cexp$: otherwise \hyperref[cost:1DLAto2NFA]{\ODLA{}\tto\TNFA} would be less than $\Cexp$ since \ONFA{}\tto\TNFA is $\Ctriv$.
\end{itemize}
\paragraph{\ODLA{}\tto\TDFA}
\begin{itemize}
	\item $\le\Cexp$: by using construction for \hyperref[cost:1DLAto1DFA]{\ODLA{}\tto\ODFA}, which is $\le\Cexp$, combined with $\Ctriv$ \ODFA{}\tto\TDFA,
	\item $\ge\Cexp$: otherwise \hyperref[cost:1DLAto2NFA]{\ODLA{}\tto\TNFA} would be less than $\Cexp$ since \TDFA{}\tto\TNFA is $\Ctriv$.
\end{itemize}


\begin{table}
	\centering
	\renewcommand{\arraystretch}{1.1}
\renewcommand{\hstdef}{.55}
\begin{tabular}{|l|l|l|p{3.1em}|l|l|p{4.3em}|}
	\hline
	~       & \ODFA              & \ONFA     & \TDFA                                        & \TNFA              & \OMOLA    & \OMODLA                                       \\ \hline
	\ODFA   & \cY                & $\Cctriv$ & $\Cctriv$                                    & $\Cctriv$          & $\Cctriv$ & $\Cctriv$                                     \\ \hline
	\ONFA   & $\Cexp$            & \cY       & \cR $\le\Cexp\hst$ $\ge\Cpoly$               & $\Cctriv$          & $\Cctriv$ & \cB $\le\Cexp\hst[1.75]$ $\ge\Cpoly$          \\ \hline
	\TDFA   & $\Cexp$            & $\Cexp$   & \cY                                          & $\Cctriv$          & $\Cctriv$ & $\Cctriv$                                     \\ \hline
	\TNFA   & $\Cexp$            & $\Cexp$   & \cR $\le\Cexp\hst$ $\ge\Cpoly$               & \cY                & $\Cctriv$ & \cB $\le\Cexp\hst[1.75]$ $\ge\Cpoly$          \\ \hline
	\OMOLA  & \rbt[.2]{$\Cdexp$} & $\Cexp$   & \cG \rbt[.2]{$\le\Cdexp\hst[.1]$} $\ge\Cexp$ & $\Cexp$            & \cY       & \cG \rbt[.2]{$\le\Cdexp\hst[1.3]$} $\ge\Cexp$ \\ \hline
	\OMODLA & $\Cexp$            & $\Cexp$   & \rbt[.1]{$O(n^3)$}                           & \rbt[.1]{$O(n^3)$} & $\Cctriv$ & \cY                                           \\ \hline
\end{tabular}
%
	\caption{Descriptional complexity of the simulations between basic regular language recognisers and once-marking $1$-limited automata, general case.}
	\label{tab:sims-om-general}
\end{table}

\Cref{tab:sims-om-general} contains the costs of simulations between basic regular language acceptors and once-marking $1$-limited automata.
Here we give a description of each nontrivial result that hasn't been explained previously:

\paragraph{\ONFA{}\tto\OMODLA}
\begin{itemize}
	\item $\le\Cexp$: by using construction for \hyperref[cost:1NFAto1DFA]{\ONFA{}\tto\ODFA}, which is $\le\Cexp$, combined with $\Ctriv$ \ODFA{}\tto\OMODLA,
	\item relaxed form of the Sakoda and Sipser conjecture with extended \TDFA.
\end{itemize}
\paragraph{\TNFA{}\tto\OMODLA}
\begin{itemize}
	\item $\le\Cexp$: by using construction for \hyperref[cost:2NFAto1DFA]{\TNFA{}\tto\ODFA}, which is $\le\Cexp$, combined with $\Ctriv$ \ODFA{}\tto\OMODLA,
	\item relaxed form of the Sakoda and Sipser conjecture with extended \TDFA.
\end{itemize}
\paragraph{\OMOLA{}\tto\ODFA}\label{cost:OM1LAto1DFA}
\begin{itemize}
	\item $\le\Cdexp$: by using the transition table construction for general \OLA presented by \triplecite{PigPis14},
	\item $\ge\Cdexp$: distinguishability arguments on witness language $K_n$ defined by \triplecite{PigPri23a}.
\end{itemize}
\paragraph{\OMOLA{}\tto\ONFA}\label{cost:OM1LAto1NFA}
\begin{itemize}
	\item $\le\Cexp$: by using the transition table construction for general \OLA presented by \triplecite{PigPis14},
	\item $\ge\Cexp$: otherwise \hyperref[cost:OM1LAto1DFA]{\OMOLA{}\tto\ODFA} would be less than $\Cdexp$ since\linebreak\hyperref[cost:1NFAto1DFA]{\ONFA{}\tto\ODFA} is $\Cexp$.
\end{itemize}
\paragraph{\OMOLA{}\tto\TDFA}
\begin{itemize}
	\item $\le\Cdexp$: by using construction for \hyperref[cost:OM1LAto1DFA]{\OMOLA{}\tto\ODFA}, which is $\le\Cdexp$, combined with $\Ctriv$ \ODFA{}\tto\TDFA,
	\item $\ge\Cexp$: otherwise \hyperref[cost:OM1LAto1DFA]{\OMOLA{}\tto\ODFA} would be less than $\Cdexp$ since\linebreak\hyperref[cost:2DFAto1DFA]{\TDFA{}\tto\ODFA} is $\Cexp$,
	\item harder form of the Sakoda and Sipser conjecture with extended \TNFA.
\end{itemize}
\paragraph{\OMOLA{}\tto\TNFA}
\begin{itemize}
	\item $\le\Cexp$: by using construction for \hyperref[cost:OM1LAto1NFA]{\OMOLA{}\tto\ONFA}, which is $\le\Cexp$, combined with $\Ctriv$ \ONFA{}\tto\TNFA,
	\item $\ge\Cexp$: otherwise \hyperref[cost:OM1LAto1DFA]{\OMOLA{}\tto\ODFA} would be less than $\Cdexp$ since\linebreak\hyperref[cost:2NFAto1DFA]{\TNFA{}\tto\ODFA} is $\Cexp$.
\end{itemize}
\paragraph{\OMODLA{}\tto\ODFA}\label{cost:OM1DLAto1DFA}
\begin{itemize}
	\item $\le\Cexp$: by using the transition table construction for general \ODLA presented by \triplecite{PigPis14},
	\item $\ge\Cexp$: since the witness language $K_n$ is recognized by a linear size \OMODLA, but cannot be recognised by a \TNFA with less than $\Cexp$ states, as proven by \triplecite{PigPri23a}.
\end{itemize}
\paragraph{\OMOLA{}\tto\OMODLA}
\begin{itemize}
	\item $\le\Cdexp$: by using construction for \hyperref[cost:OM1LAto1DFA]{\OMOLA{}\tto\ODFA}, which is $\le\Cdexp$, combined with $\Ctriv$ \ODFA{}\tto\OMODLA,
	\item $\ge\Cexp$: otherwise \hyperref[cost:OM1LAto1DFA]{\OMOLA{}\tto\ODFA} would be less than $\Cdexp$ since \hyperref[cost:OM1DLAto1DFA]{\OMODLA{}\tto\ODFA} is $\Cexp$,
	\item variation of the Sakoda and Sipser problem with both extended \TNFA and \TDFA.
\end{itemize}
% TODO: once the proof has been written, manage its reference here
\paragraph{\OMODLA{}\tto\ONFA}
\begin{itemize}
	\item $\le\Cexp$: by using construction for \hyperref[cost:OM1DLAto1DFA]{\OMODLA{}\tto\ODFA}, which is $\le\Cexp$, combined with $\Ctriv$ \ODFA{}\tto\ONFA,
	\item \textbf{previously mistakenly considered poly},
	\item $\ge\Cexp$: since the witness language $D_n:=\set{ww\mid w\in\set{a,b}^n}$ is recognized by a linear size \OMODLA, but cannot be recognised by a \ONFA with less than $\Cexp$ states \textbf{(proven by us by fooling set)}.
\end{itemize}
\paragraph{\OMODLA{}\tto\TDFA}\label{cost:OM1DLAto2DFA}
\begin{itemize}
	\item $\le O(n^3)$: by using the simulation via computation trees given by \triplecite{PigPri23a}.
\end{itemize}
\paragraph{\OMODLA{}\tto\TNFA}
\begin{itemize}
	\item $\le O(n^3)$: by using construction for \hyperref[cost:OM1DLAto2DFA]{\OMODLA{}\tto\TDFA}, which is $\le O(n^3)$, combined with $\Ctriv$ \TDFA{}\tto\TNFA.
\end{itemize}



\section{The unary case}\label{sec:prevsims-unary}

\begin{table}
	\centering
	\renewcommand{\arraystretch}{1.2}
\begin{tabular}{|l|l|l|p{4.3em}|l|}
	\hline
	~     & \ODFA           & \ONFA           & \TDFA                                            & \TNFA     \\ \hline
	\ODFA & \cY             & $\Cctriv$       & $\Cctriv$                                        & $\Cctriv$ \\ \hline
	\ONFA & \rbt{$\CsubEq$} & \cY             & \cR $\Theta(n^2)$                                & $\Cctriv$ \\ \hline
	\TDFA & \rbt{$\CsubEq$} & \rbt{$\CsubEq$} & \cY                                              & $\Cctriv$ \\ \hline
	\TNFA & \rbt{$\CsubEq$} & \rbt{$\CsubEq$} & \cR \rbt[.4]{$\le\Csubln$} \rbt[.3]{$\ge\Cpoly$} & \cY       \\ \hline
\end{tabular}
%
	\caption{Descriptional complexity of the simulations between basic regular language recognisers, unary case.}
	\label{tab:sims-core-unary}
\end{table}

\Cref{tab:sims-core-unary} contains the costs of simulations between basic regular language acceptors for unary languages.
Many of these costs, mostly found by pioneer of unary descriptional complexity Marek Chrobak, are given by the Landau function:
\begin{equation}
	F(n) := \max\set{\lcm(x_1,\dots,x_k)\mid x_1+\dots+x_k=n} \text,
\end{equation}
where $n\in\N$.
The problem of approximating this function is known as Landau's problem.
For the sake of our usage of the function, the following approximation, which was found by Landau himself in the early twentieth century \cite{Lan03,Lan09}, is sufficient:
\begin{equation}
	F(n) = \CsubEq \text.
\end{equation}

We refer to functions such as the one above, which can be expressed in the form $e^{o(n)}$, as "subexponential", which serve as a midway between the polynomial and exponential complexities.
Here we give a description of each nontrivial result in \Cref{tab:sims-core-unary}:

\paragraph{\ONFA{}\tto\ODFA}
\begin{itemize}
	\item $\le\CsubLe$: proven by \triplecite{Chr86},
	\item $\ge\CsubGe$: proven by \triplecite{Chr86}.
\end{itemize}
\paragraph{\TDFA{}\tto\ODFA}\label{cost:2DFAto1DFAu}
\begin{itemize}
	\item $\le\CsubLe$: proven by \triplecite{Chr86},
	\item $\ge\CsubGe$: proven by \triplecite{Chr86}.
\end{itemize}
\paragraph{\TDFA{}\tto\ONFA}\label{cost:2DFAto1NFAu}
\begin{itemize}
	% \item $\le\CsubLe$: by using construction for \hyperref[cost:2DFAto1DFAu]{\TDFA{}\tto\ODFA}, which is $\le\CsubLe$, combined with $\Ctriv$ \ODFA{}\tto\ONFA,
	\item $\le\CsubLe$: by using construction for \hyperref[cost:2DFAto1DFAu]{\TDFA{}\tto\ODFA}, which is at most $\CsubLe$, combined with $\Ctriv$ \ODFA{}\tto\ONFA,
	\item $\ge\CsubGe$: proven by \triplecite{Chr86} (shown as an improvement of the lower bound of \hyperref[cost:2DFAto1DFAu]{\TDFA{}\tto\ODFA}).
\end{itemize}
\paragraph{\ONFA{}\tto\TDFA}\label{cost:1NFAto2DFAu}
\begin{itemize}
	\item $\le O(n^2)$: proven by \triplecite{Chr86},
	\item $\ge \Omega(n^2)$: proven by \triplecite{Chr86},
	\item solved Sakoda and Sipser problem for the unary case.
\end{itemize}
\paragraph{\TNFA{}\tto\ODFA}\label{cost:2NFAto1DFAu}
\begin{itemize}
	\item $\le\CsubLe$: using construction presented by \triplecite{MerPig01},
	\item $\ge\CsubGe$: otherwise \hyperref[cost:2DFAto1DFAu]{\TDFA{}\tto\ODFA} would be less than $\CsubLe$ since \TDFA{}\tto\TNFA is $\Ctriv$.
\end{itemize}
\paragraph{\TNFA{}\tto\ONFA}
\begin{itemize}
	\item $\le\CsubLe$: by using construction for \hyperref[cost:2NFAto1DFAu]{\TNFA{}\tto\ODFA}, which is $\le\Cexp$, combined with $\Ctriv$ \ODFA{}\tto\ONFA,
	\item $\ge\CsubGe$: otherwise \hyperref[cost:2DFAto1NFAu]{\TDFA{}\tto\ONFA} would be less than $\CsubLe$ since \TDFA{}\tto\TNFA is $\Ctriv$.
\end{itemize}
\paragraph{\TNFA{}\tto\TDFA}\label{cost:2NFAto2DFAu}
\begin{itemize}
	\item $\le e^{O(\ln^2n)}$: proven by \triplecite{GefMer+03} (and mentioned by \triplecite{Pig15}),
	\item (already proven $\le\CsubLe$ by using construction for \hyperref[cost:2NFAto1DFAu]{\TNFA{}\tto\ODFA}, which is $\le\CsubLe$, combined with $\Ctriv$ \ODFA{}\tto\TDFA),
	\item the only unsolved Sakoda and Sipser problem for the unary case (\hyperref[cost:1NFAto2DFAu]{\ONFA{}\tto\TDFA} is solved).
\end{itemize}


\begin{table}
	\centering
	\renewcommand{\arraystretch}{1.2}
\renewcommand{\hstdef}{1.75}
\begin{tabular}{|l|p{4.2em}|l|p{4.3em}|l|l|p{4.3em}|}
	\hline
	~     & \ODFA                            & \ONFA           & \TDFA                                            & \TNFA     & \OLA      & \ODLA                                            \\ \hline
	\ODFA & \cY                              & $\Cctriv$       & $\Cctriv$                                        & $\Cctriv$ & $\Cctriv$ & $\Cctriv$                                        \\ \hline
	\ONFA & \rbt{$\CsubEq$}                  & \cY             & \cR $\Theta(n^2)$                                & $\Cctriv$ & $\Cctriv$ & \cB $O(n^2)$                                     \\ \hline
	\TDFA & \rbt{$\CsubEq$}                  & \rbt{$\CsubEq$} & \cY                                              & $\Cctriv$ & $\Cctriv$ & $\Cctriv$                                        \\ \hline
	\TNFA & \rbt{$\CsubEq$}                  & \rbt{$\CsubEq$} & \cR \rbt[.3]{$\le\Csubln$} \rbt[.2]{$\ge\Cpoly$} & \cY       & $\Cctriv$ & \cB \rbt[.3]{$\le\Csubln$} \rbt[.2]{$\ge\Cpoly$} \\ \hline
	\OLA  & $\le\Cdexp\hst[1.15]$ $\ge\Cexp$ & $\Cexp$         & \cG $\Cexp$                                      & $\Cexp$   & \cY       & \cG $\le\Cexp\hst$ $\ge\Cpoly$                   \\ \hline
	\ODLA & $\Cexp$                          & $\Cexp$         & $\Cexp$                                          & $\Cexp$   & $\Cctriv$ & \cY                                              \\ \hline
\end{tabular}
%
	\caption{Descriptional complexity of the simulations between basic regular language recognisers and $1$-limited automata, unary case.}
	\label{tab:sims-1la-unary}
\end{table}

\Cref{tab:sims-1la-unary} contains the costs of simulations between basic regular language acceptors and $1$-limited automata, for unary languages.
Here we give a description of each nontrivial result that hasn't been explained previously:

\paragraph{\ONFA{}\tto\ODLA}\label{cost:1NFAto1DLAu}
\begin{itemize}
	\item $\le O(n^2)$: by using construction for \hyperref[cost:1NFAto2DFAu]{\ONFA{}\tto\TDFA}, which is $\le O(n^2)$, combined with $\Ctriv$ \TDFA{}\tto\ODLA,
	\item relaxed form of the Sakoda and Sipser conjecture with extended \TDFA.
\end{itemize}
\paragraph{\TNFA{}\tto\ODLA}
\begin{itemize}
	\item $\le e^{O(\ln^2n)}$: by using construction for \hyperref[cost:2NFAto2DFAu]{\TNFA{}\tto\TDFA}, which is $\le e^{O(\ln^2n)}$, combined with $\Ctriv$ \TDFA{}\tto\ODLA,
	\item relaxed form of the Sakoda and Sipser conjecture with extended \TDFA.
\end{itemize}
\paragraph{\ODLA{}\tto\ODFA}\label{cost:1DLAto1DFAu}
\begin{itemize}
	\item $\le\Cexp$: by using the transition table construction for the \hyperref[cost:1DLAto1DFA]{general case} \cite{PigPis14},
	\item $\ge\Cexp$: otherwise \hyperref[cost:1DLAto2NFAu]{\ODLA{}\tto\TNFA} would be less than $\Cexp$ since \ODFA{}\tto\TNFA is $\Ctriv$ (see lower bound below).
\end{itemize}
\paragraph{\ODLA{}\tto\TNFA}\label{cost:1DLAto2NFAu}
\begin{itemize}
	\item $\le\Cexp$: by using construction for \hyperref[cost:1DLAto1DFAu]{\ODLA{}\tto\ODFA}, which is $\le\Cexp$, combined with $\Ctriv$ \ODFA{}\tto\TNFA,
	\item $\ge\Cexp$: since the witness language $\set{a^{2^n}}\star$ is recognized by the linear size \ODLA presented by \triplecite{PigPri19}, but cannot be recognised by a \TNFA with less than $\Cexp$ states, as proven by \triplecite{MerPig00},
	\item (previously proven $\ge\CsubGe$ by \triplecite{KutPig+18}).
\end{itemize}
\paragraph{\ODLA{}\tto\ONFA}\label{cost:1LAto1NFAu}
\begin{itemize}
	\item $\le\Cexp$: by using construction for \hyperref[cost:1DLAto1DFAu]{\ODLA{}\tto\ODFA}, which is $\le\Cexp$, combined with $\Ctriv$ \ODFA{}\tto\ONFA,
	\item $\ge\Cexp$: otherwise \hyperref[cost:1DLAto2NFAu]{\ODLA{}\tto\TNFA} would be less than $\Cexp$ since \ONFA{}\tto\TNFA is $\Ctriv$.
\end{itemize}
\paragraph{\ODLA{}\tto\TDFA}\label{cost:1DLAto2DFAu}
\begin{itemize}
	\item $\le\Cexp$: by using construction for \hyperref[cost:1DLAto1DFAu]{\ODLA{}\tto\ODFA}, which is $\le\Cexp$, combined with $\Ctriv$ \ODFA{}\tto\TDFA,
	\item $\ge\Cexp$: otherwise \hyperref[cost:1DLAto2NFAu]{\ODLA{}\tto\TNFA} would be less than $\Cexp$ since \TDFA{}\tto\TNFA is $\Ctriv$.
\end{itemize}
\paragraph{\OLA{}\tto\ODFA}
\begin{itemize}
	\item $\le\Cdexp$: by using the transition table construction for the \hyperref[cost:1LAto1DFA]{general case} \cite{PigPis14},
	\item $\ge\Cexp$: otherwise \hyperref[cost:1DLAto1DFAu]{\ODLA{}\tto\ODFA} would be less than $\Cexp$ since \ODLA{}\tto\OLA is $\Ctriv$.
\end{itemize}
\paragraph{\OLA{}\tto\ONFA}\label{cost:1DLAto1NFAu}
\begin{itemize}
	\item $\le\Cexp$: by using the transition table construction for the \hyperref[cost:1LAto1NFA]{general case} \cite{PigPis14},
	\item $\ge\Cexp$: otherwise \hyperref[cost:1DLAto1NFAu]{\ODLA{}\tto\ONFA} would be less than $\Cexp$ since \ODLA{}\tto\OLA is $\Ctriv$.
\end{itemize}
\paragraph{\OLA{}\tto\TDFA}
\begin{itemize}
	\item $\le\Cexp$: by using construction for \hyperref[cost:1LAto1NFAu]{\OLA{}\tto\ONFA}, which is $\le\Cexp$, combined with $O(n^2)$ \hyperref[cost:1NFAto2DFAu]{\ONFA{}\tto\TDFA},
	\item $\ge\Cexp$: otherwise \hyperref[cost:1DLAto2DFAu]{\ODLA{}\tto\TDFA} would be less than $\Cexp$ since \ODLA{}\tto\OLA is $\Ctriv$,
	\item solved harder form of the Sakoda and Sipser conjecture with extended \TNFA.
\end{itemize}
\paragraph{\OLA{}\tto\TNFA}
\begin{itemize}
	\item $\le\Cexp$: by using construction for \hyperref[cost:1LAto1NFAu]{\OLA{}\tto\ONFA}, which is $\le\Cexp$, combined with $\Ctriv$ \ONFA{}\tto\TNFA,
	\item $\ge\Cexp$: otherwise \hyperref[cost:1DLAto2NFAu]{\ODLA{}\tto\TNFA} would be less than $\Cexp$ since \ODLA{}\tto\OLA is $\Ctriv$.
\end{itemize}
\paragraph{\OLA{}\tto\ODLA}
\begin{itemize}
	\item $\le\Cexp$: by using construction for \hyperref[cost:1LAto1NFAu]{\OLA{}\tto\ONFA}, which is $\le\Cexp$, combined with $O(n^2)$ \hyperref[cost:1NFAto1DLAu]{\ONFA{}\tto\ODLA},
	\item variation of the Sakoda and Sipser problem with both extended \TNFA and \TDFA.
\end{itemize}


\begin{table}
	\makebox[\textwidth]{%
	\begin{minipage}{\dimexpr\textwidth+\oddsidemargin+\evensidemargin\relax}
		\centering
		\renewcommand{\arraystretch}{1.2}
		\renewcommand{\hstdef}{1.75}
		\begin{tabular}{|l|p{5.4em}|p{5.4em}|p{4.3em}|p{2.9em}|l|p{4.3em}|}
			\hline
			~       & \ODFA                              & \ONFA                             & \TDFA                                            & \TNFA                           & \OMOLA    & \OMODLA                                          \\ \hline
			\ODFA   & \cY                                & $\Cctriv$                         & $\Cctriv$                                        & $\Cctriv$                       & $\Cctriv$ & $\Cctriv$                                        \\ \hline
			\ONFA   & \rbt{$\CsubEq$}                    & \cY                               & \cR $\Theta(n^2)$                                & $\Cctriv$                       & $\Cctriv$ & \cB $O(n^2)$                                     \\ \hline
			\TDFA   & \rbt{$\CsubEq$}                    & \rbt{$\CsubEq$}                   & \cY                                              & $\Cctriv$                       & $\Cctriv$ & $\Cctriv$                                        \\ \hline
			\TNFA   & \rbt{$\CsubEq$}                    & \rbt{$\CsubEq$}                   & \cR \rbt[.4]{$\le\Csubln$} \rbt[.3]{$\ge\Cpoly$} & \cY                             & $\Cctriv$ & \cB \rbt[.4]{$\le\Csubln$} \rbt[.3]{$\ge\Cpoly$} \\ \hline
			\OMOLA  & $\le\Cdexp\hst[2.35]$ $\ge\CsubGe$ & $\le\Cexp\hst[2.85]$ $\ge\CsubGe$ & \cG $\le\Cexp\hst$ $\ge\Cpoly$                   & $\le\Cexp\hst[.35]$ $\ge\Cpoly$ & \cY       & \cG $\le\Cexp\hst$ $\ge\Cpoly$                   \\ \hline
			\OMODLA & $\le\Cexp\hst[2.85]$ $\ge\CsubGe$  & $\le\Cexp\hst[2.85]$ $\ge\CsubGe$ & $O(n^3)$                                         & $O(n^3)$                        & $\Cctriv$ & \cY                                              \\ \hline
		\end{tabular}
	\end{minipage}%
}
%
	\caption{Descriptional complexity of the simulations between basic regular language recognisers and once-marking $1$-limited automata, unary case.}
	\label{tab:sims-om-unary}
\end{table}

\Cref{tab:sims-om-unary} contains the costs of simulations between basic regular language acceptors and once-marking $1$-limited automata, in the unary case.
Here we give a description of each nontrivial result that hasn't been explained previously:

\paragraph{\ONFA{}\tto\OMODLA}\label{cost:1NFAtoOM1DLAu}
\begin{itemize}
	\item $\le O(n^2)$: by using construction for \hyperref[cost:1NFAto2DFAu]{\ONFA{}\tto\TDFA}, which is $\le O(n^2)$, combined with $\Ctriv$ \TDFA{}\tto\OMODLA.
\end{itemize}
\paragraph{\TNFA{}\tto\OMODLA}
\begin{itemize}
	\item $\le e^{O(\ln^2n)}$: by using construction for \hyperref[cost:2NFAto2DFAu]{\TNFA{}\tto\TDFA}, which is $\le e^{O(\ln^2n)}$, combined with $\Ctriv$ \TDFA{}\tto\OMODLA,
	\item relaxed form of the Sakoda and Sipser conjecture with extended \TDFA.
\end{itemize}
\paragraph{\OMOLA{}\tto\ODFA}
\begin{itemize}
	\item $\le\Cdexp$: by using the transition table construction for general \OLA presented by \triplecite{PigPis14},
	\item $\ge\CsubGe$: otherwise \hyperref[cost:2DFAto1DFAu]{\TDFA{}\tto\ODFA} would be less than $\CsubLe$ since \TDFA{}\tto\OMOLA is $\Ctriv$.
\end{itemize}
\paragraph{\OMOLA{}\tto\ONFA}\label{cost:OM1LAto1NFAu}
\begin{itemize}
	\item $\le\Cexp$: by using the transition table construction for general \OLA presented by \triplecite{PigPis14},
	\item $\ge\CsubGe$: otherwise \hyperref[cost:2DFAto1NFAu]{\TDFA{}\tto\ONFA} would be less than $\CsubLe$ since \TDFA{}\tto\OMOLA is $\Ctriv$.
\end{itemize}
\paragraph{\OMOLA{}\tto\TDFA}
\begin{itemize}
	\item $\le\Cexp$: by using construction for \hyperref[cost:OM1LAto1NFAu]{\OMOLA{}\tto\ONFA}, which is $\le\Cexp$, combined with $O(n^2)$ \ONFA{}\tto\TDFA.
\end{itemize}
\paragraph{\OMOLA{}\tto\TNFA}
\begin{itemize}
	\item $\le\Cexp$: by using construction for \hyperref[cost:OM1LAto1NFAu]{\OMOLA{}\tto\ONFA}, which is $\le\Cexp$, combined with $\Ctriv$ \ONFA{}\tto\TNFA.
\end{itemize}
\paragraph{\OMOLA{}\tto\OMODLA}
\begin{itemize}
	\item $\le\Cexp$: by using construction for \hyperref[cost:OM1LAto1NFAu]{\OMOLA{}\tto\ONFA}, which is $\le\Cexp$, combined with $O(n^2)$ \hyperref[cost:1NFAtoOM1DLAu]{\ONFA{}\tto\OMODLA}.
\end{itemize}
\paragraph{\OMODLA{}\tto\ODFA}\label{cost:OM1DLAto1DFAu}
\begin{itemize}
	\item $\le\Cexp$: by using the transition table construction for general \ODLA presented by \triplecite{PigPis14},
	\item $\ge\CsubGe$: otherwise \hyperref[cost:2DFAto1DFAu]{\TDFA{}\tto\ODFA} would be less than $\CsubLe$ since \TDFA{}\tto\OMODLA is $\Ctriv$.
\end{itemize}
\paragraph{\OMODLA{}\tto\ONFA}
\begin{itemize}
	\item $\le\Cexp$: by using construction for \hyperref[cost:OM1DLAto1DFAu]{\OMODLA{}\tto\ODFA}, which is $\le\Cexp$, combined with $\Ctriv$ \ODFA{}\tto\ONFA,
	\item $\ge\CsubGe$: otherwise \hyperref[cost:2DFAto1NFAu]{\TDFA{}\tto\ONFA} would be less than $\CsubLe$ since \TDFA{}\tto\OMODLA is $\Ctriv$.
\end{itemize}
\paragraph{\OMODLA{}\tto\TDFA}\label{cost:OM1DLAto2DFAu}
\begin{itemize}
	\item $\le O(n^3)$: by using the simulation via computation trees for general \OMODLA presented by \triplecite{PigPri23a}.
\end{itemize}
\paragraph{\OMODLA{}\tto\TNFA}
\begin{itemize}
	\item $\le O(n^3)$: by using construction for \hyperref[cost:OM1DLAto2DFAu]{\OMODLA{}\tto\TDFA}, which is $\le O(n^3)$, combined with $\Ctriv$ \TDFA{}\tto\TNFA.
\end{itemize}
