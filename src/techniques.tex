\chapter{From \emph{two-way} to \emph{one-way} automata}\label{ch:techniques}



% TODO
% TODO: specify that the first part is heavily based on HopUll
\section{Introduction}
\dots


\section{Crossing sequences}\label{sec:crossseq2DFA}
% TODO: add example (with pic)?
Given a two-way machine and one of its possible computations over a certain input, consider a tape cell and one of the two \emph{boundaries} at its sides.
A crossing sequence $[q_1,q_2,\dots]$ is the sequence of states that the machine enters as the head crosses such boundary, \ie it enters or leaves the cell on the chosen side.

In general, because of loops and nondeterminism, a crossing sequence may be of infinite length.
However, we shall see that we may restrict our interest to some finite set of finite sequences, depending on the model under consideration.

As two-way machines start their computations on the left end-marker, the first state of every nonempty crossing sequence is always entered with a move to the right, indicating the first visit to the new cell, while subsequent crossings are in alternate directions.
Therefore, the states at odd indices of every crossing sequence are entered after passing the boundary to the right, while the ones at even indices are entered after passing the boundary to the left:
\begin{fact}\label{fact:crossing-parity}
	In a crossing sequence, states in odd positions are entered after moves to the right, while states in even positions are entered after moves to the left.
\end{fact}

Furthermore, since in the models we consider the input is accepted by crossing the right boundary of the tape\footnote{%
	Other models can be easily be converted to adhere to this behavior: if a machine accepts when a transition is undefined in a final state, that transition can be defined to bring to a new final state, which moves the head to the right until it passes the right boundary of the tape.}% TODO: this consideration may be moved in the chapter where models are defined. TBD
, the last crossing of each boundary is towards the right, \ie in an odd position:
\begin{fact}\label{fact:crossing-length}
	Every crossing sequence of an accepting computation has odd length.
\end{fact}


\subsection{2-way deterministic finite automata}
We now consider the specific case of two-way deterministic finite automata (\TDFAs).
If a \TDFA accepts an input word $w$, the computation on $w$ will never enter the same state twice while passing the same boundary in the same direction, otherwise the machine, being deterministic, would run in an infinite loop and thus not accept.
In terms of crossing sequences:
\begin{fact}\label{fact:crossing-2DFA-parity}
	No crossing sequence of an accepting computation of a \TDFA may contain the same state in two positions of the same parity.
\end{fact}

We are now ready to define \emph{valid crossing sequences} for \TDFAs:
\begin{defn}
	Given a \TDFA $A$, a crossing sequence over the states of $A$ is called \emph{valid} if and only if:
	\begin{itemize}
		\item Its length is odd and bounded by $2n$, where $n$ is the number of states of $A$.
		\item No state is repeated in two positions of the same parity.
	\end{itemize}
\end{defn}

Notice that, following \Cref{fact:crossing-parity,fact:crossing-length,fact:crossing-2DFA-parity}, the following holds:
\begin{fact}
	All crossing sequences of accepting computations of a \TDFA are valid.
\end{fact}

Obviously, since the length of valid crossing sequences is bounded, their number is finite.
In order to calculate this number, consider the following sequence of choices:
\begin{enumerate}
	\item \label{itm:num-crosseq-1} the first and second states are chosen among $n$,
	\item the third and fourth states are chosen among $n-1$, since the states chosen at the previous step can no longer appear in positions of the same parity,
	\item[$i$.] in general, the $(2i-1)$-th and $2i$-th states are chosen among $n-i+1$.
\end{enumerate}
By the \emph{fundamental principle of counting}, the total number of choices is obtained by multiplying the choices at each step.
Therefore, the number of possible crossing sequences of length $2l-1$ (with $l\in\set{1,\dots,n}$) is given by:
\begin{equation*}
	n \cdot n \cdot (n-1) \cdot (n-1) \cdots \underbrace{(n-l+2)}_{(2l-3)\text{-th}} \cdot \underbrace{(n-l+2)}_{(2l-2)\text{-th}} \cdot \underbrace{(n-l+1)}_{(2l-1)\text{-th}} \text,
\end{equation*}
which, by grouping odd- and even-numbered choices, we can write in the form
\begin{equation*}
	= \underbrace{n \cdot (n-1) \cdots (n-l+2) \cdot (n-l+1)}_{=\dfrac{n!}{(n-l)!}} \; \cdot \; \underbrace{n \cdot (n-1) \cdots (n-l+2)}_{=\dfrac{n!}{(n-l+1)!}} \\
\end{equation*}
Therefore, including all possible lengths, the total number of valid crossing sequences is
\begin{align*}
	\sum_{l=1}^n \left(\frac{n!}{(n-l)!}\cdot\frac{n!}{(n-l+1)!}\right) \approx (n!)^2 \approx n^{2n} = 2^{2n \log_2 n} \text.
\end{align*}

\begin{fact}\label{fact:crossing-2DFA-num}
	The number of valid crossing sequences of a \TDFA is exponential in the number of states.
\end{fact}


\subsubsection{Matching crossing sequences}
In an accepting computation, the two crossing sequences at the boundaries of each cell are \emph{compatible}, meaning their states are the result of valid transitions given the cell's content.
In general, given a tape cell and two crossing sequences $[q_1,\dots,q_k]$ and $[p_1,\dots,p_l]$ for its left and right boundary respectively, we can check for \emph{local} compatibility starting in state $q_1$ and proceeding as follows.
\begin{itemize}
	\item If the head moves left from the cell into state $q_i$, assume the automaton going back to the cell in state $p_{i+1}$ and perform a move, checking that it enters the next state of the crossing sequence of the correct boundary.
	\item If the head moves to the right in state $p_i$, assume the automaton going back to the cell in state $p_{i+1}$ and perform a move as in the previous case.
\end{itemize}

We then say that the sequence $[q_1,\dots,q_k]$ \emph{right-matches} the sequence $[p_1,\dots,p_l]$ with respect to a symbol $\sigma$ when they are locally compatible starting with a transition to the right on the cell containing $\sigma$ in state $q_1$ and ending to its right.
We say that $[q_1,\dots,q_k]$ \emph{left-matches} $[p_1,\dots,p_l]$ when the initial transition is to the left in state $p_1$, while still ending to the right of the cell.

For a visual representation of the formal definition see \Cref{fig:2DFA-crossmatch}.
\begin{defn}
	Given a \TDFA $A=(Q,\Sigma,\delta,q_I,F)$ we define left- and right-matching pairs of crossing sequences with respect to a symbol $\sigma\in\Sigma$ as follows:
	\begin{rules}
		\item \label{itm:crossmatch2DFA-1} The empty sequence $[~]$ right-matches and left-matches itself with respect to~$\sigma$.
		\item \label{itm:crossmatch2DFA-2} If $[q_3,\dots,q_k]$ right-matches $[p_1,\dots,p_l]$ with respect to~$\sigma$ and $\delta(q_1,\sigma)=(q_2,\tl)$, then $[q_1,\dots,q_k]$ right-matches $[p_1,\dots,p_l]$ with respect to~$\sigma$.
		The computation enters the cell from the left in state $q_1$, turns around right away leaving it to the left in $q_2$, and eventually comes back to it in $q_3$, to which we apply induction.
		\item \label{itm:crossmatch2DFA-3} If $[q_2,\dots,q_k]$ left-matches $[p_2,\dots,p_l]$ with respect to~$\sigma$ and $\delta(q_1,\sigma)=(p_1,\tr)$, then $[q_1,\dots,q_k]$ right-matches $[p_1,\dots,p_l]$ with respect to~$\sigma$.
		The computation enters the cell from the right in state $p_1$, immediately leaves it to the left in $q_1$, and eventually comes back to it in $q_2$, to which we apply induction.
		\item \label{itm:crossmatch2DFA-4} Similarly to \Cref{itm:crossmatch2DFA-2}, if $[q_1,\dots,q_k]$ left-matches $[p_3,\dots,p_l]$ with respect to~$\sigma$ and $\delta(p_1,\sigma)=(p_2,\tr)$, then $[q_1,\dots,q_k]$ left-matches $[p_1,\dots,p_l]$ with respect to~$\sigma$.
		\item \label{itm:crossmatch2DFA-5} Similarly to \Cref{itm:crossmatch2DFA-3}, if $[q_2,\dots,q_k]$ right-matches $[p_2,\dots,p_l]$ with respect to~$\sigma$ and $\delta(p_1,\sigma)=(q_1,\tl)$, then $[q_1,\dots,q_k]$ left-matches $[p_1,\dots,p_l]$ with respect to~$\sigma$.
	\end{rules}
\end{defn}
Note that two crossing sequences need not be valid in order to match (\eg the empty sequence).

\begin{figure}
	\centering
	\begin{subfigure}{0.246\textwidth}
		\centering
		\begin{tikzpicture}[tapeseg/.append style={minimum width=2.5em}]
	\node[cell,thick] (sigma) {$\sigma$};
	\draw[thick,dashed] (sigma.north west) -- ++(-6mm,0) (sigma.south west) -- ++(-6mm,0);
	\draw[thick,dashed] (sigma.north east) -- ++(+6mm,0) (sigma.south east) -- ++(+6mm,0);
	\draw[dashed,shorten <=.1cm,thin] (sigma.south west) -- ++(0cm,-28mm);
	\draw[dashed,shorten <=.1cm,thin] (sigma.south east) -- ++(0cm,-28mm);

	\node[tapeseg,node distance=1pt] (leftalign) at ([xshift=-1.5mm]sigma) {};
	\node[tapeseg,node distance=1pt] (leftleftalign) at ([xshift=-7mm]sigma) {};
	\node[tapeseg,node distance=1pt] (rightrightalign) at ([xshift=7mm]sigma) {};

	\node (q1) [below=.5mm of leftalign] {$q_1$};
	\node (q2) [below=4mm of leftleftalign] {$q_2$};
	\draw[transition] ([shift={(-5mm,2mm)}]q1.south) -- (q1.south);
	\draw[transition] (q1.south) -- (q2.south);

	\draw[dotted,thin] ([shift={(-6mm,-13mm)}]sigma.west) -- ([shift={(6mm,-13mm)}]sigma.east);

	\node (q3) [below=5mm of q1] {$q_3$};
	\draw[transition,smalldashed] (q2.south) .. controls +(-.5,-.3) .. (q3.south);
	\node (qdots) [below=-2 mm of q3] {$\vdots$};
	\node (qk) [below=-1mm of qdots] {$q_k$};
	\draw[transition] ([shift={(-5mm,2mm)}]qk.south) -- (qk.south);

	\node (p1) [below=11.5mm of rightrightalign] {$p_1$};
	\node (pdots) [below=-2mm of p1] {$\vdots$};
	\node (pl) [below=-1mm of pdots] {$p_l$};
	\draw[transition] ([shift={(-5mm,2mm)}]p1.south) -- (p1.south);
	\draw[transition] ([shift={(-5mm,2mm)}]pl.south) -- (pl.south);

	\scriptsize
	\draw[brace] ([shift={(.6mm,13.5mm)}]pl.east) -- node[right=2pt] {r.m.} ([shift={(.6mm,-3.5mm)}]pl.east);
\end{tikzpicture}

		\caption*{\ref{itm:crossmatch2DFA-2} r.m.\tto r.m.}
	\end{subfigure}
	\hfill
	\begin{subfigure}{0.242\textwidth}
		\centering
		\begin{tikzpicture}[tapeseg/.append style={minimum width=2.5em}]
	\node[cell,thick] (sigma) {$\sigma$};
	\draw[thick,dashed] (sigma.north west) -- ++(-6mm,0) (sigma.south west) -- ++(-6mm,0);
	\draw[thick,dashed] (sigma.north east) -- ++(+6mm,0) (sigma.south east) -- ++(+6mm,0);
	\draw[dashed,shorten <=.1cm,thin] (sigma.south west) -- ++(0cm,-28mm);
	\draw[dashed,shorten <=.1cm,thin] (sigma.south east) -- ++(0cm,-28mm);

	\node[tapeseg,node distance=1pt] (leftalign) at ([xshift=-1.5mm]sigma) {};
	\node[tapeseg,node distance=1pt] (leftleftalign) at ([xshift=-7mm]sigma) {};
	\node[tapeseg,node distance=1pt] (rightalign) at ([xshift=1.5mm]sigma) {};
	\node[tapeseg,node distance=1pt] (rightrightalign) at ([xshift=7mm]sigma) {};

	\node (q1) [below=.5mm of leftalign] {$q_1$};
	\node (p1) [below=4mm of rightrightalign] {$p_1$};
	\draw[transition] ([shift={(-5mm,2mm)}]q1.south) -- (q1.south);
	\draw[transition] (q1.south) -- (p1.south);

	\draw[dotted,thin] ([shift={(-6mm,-13mm)}]sigma.west) -- ([shift={(6mm,-13mm)}]sigma.east);

	\node (p2) [below=10.5mm of rightalign] {$p_2$};
	\draw[transition,smalldashed] (p1.south) .. controls +(.5,-.3) .. (p2.south);
	\node (pdots) [below=5.5mm of p1] {$\vdots$};
	\node (pl) [below=-1mm of pdots] {$p_l$};
	\draw[transition] ([shift={(-5mm,2mm)}]pl.south) -- (pl.south);

	\node (q2) [below=12mm of leftleftalign] {$q_2$};
	\node (qdots) [below=8mm of q1] {$\vdots$};
	\node (qk) [below=-1mm of qdots] {$q_k$};
	\draw[transition] ([shift={(5mm,2mm)}]q2.south) -- (q2.south);
	\draw[transition] ([shift={(-5mm,2mm)}]qk.south) -- (qk.south);

	\scriptsize
	\draw[brace] ([shift={(1mm,13.5mm)}]pl.east) -- node[right=2pt] {l.m.} ([shift={(1mm,-3.5mm)}]pl.east);
\end{tikzpicture}

		\caption*{\ref{itm:crossmatch2DFA-3} l.m.\tto r.m.}
	\end{subfigure}
	\hfill
	\begin{subfigure}{0.242\textwidth}
		\centering
		
\begin{tikzpicture}[tapeseg/.append style={minimum width=2.5em}]
	\node[cell,thick] (sigma) {$\sigma$};
	\draw[thick,dashed] (sigma.north west) -- ++(-6mm,0) (sigma.south west) -- ++(-6mm,0);
	\draw[thick,dashed] (sigma.north east) -- ++(+6mm,0) (sigma.south east) -- ++(+6mm,0);
	\draw[dashed,shorten <=.1cm,thin] (sigma.south west) -- ++(0cm,-28mm);
	\draw[dashed,shorten <=.1cm,thin] (sigma.south east) -- ++(0cm,-28mm);

	\node[tapeseg,node distance=1pt] (leftalign) at ([xshift=-1.5mm]sigma) {};
	\node[tapeseg,node distance=1pt] (leftleftalign) at ([xshift=-7mm]sigma) {};
	\node[tapeseg,node distance=1pt] (rightalign) at ([xshift=1.5mm]sigma) {};
	\node[tapeseg,node distance=1pt] (rightrightalign) at ([xshift=7mm]sigma) {};

	\node (p1) [below=.5mm of rightalign] {$p_1$};
	\node (p2) [below=4mm of rightrightalign] {$p_2$};
	\draw[transition] ([shift={(5mm,2mm)}]p1.south) -- (p1.south);
	\draw[transition] (p1.south) -- (p2.south);

	\draw[dotted,thin] ([shift={(-6mm,-13mm)}]sigma.west) -- ([shift={(6mm,-13mm)}]sigma.east);

	\node (p3) [below=10.5mm of rightalign] {$p_3$};
	\draw[transition,smalldashed] (p2.south) .. controls +(.5,-.3) .. (p3.south);
	\node (pdots) [below=5.5mm of p2] {$\vdots$};
	\node (pl) [below=-1mm of pdots] {$p_l$};
	\draw[transition] ([shift={(-5mm,2mm)}]pl.south) -- (pl.south);

	\node (q1) [below=12mm of leftleftalign] {$q_1$};
	\node (qdots) [below=14mm of leftalign] {$\vdots$};
	\node (qk) [below=-1mm of qdots] {$q_k$};
	\draw[transition] ([shift={(5mm,2mm)}]q1.south) -- (q1.south);
	\draw[transition] ([shift={(-5mm,2mm)}]qk.south) -- (qk.south);

	\scriptsize
	\draw[brace] ([shift={(1mm,13.5mm)}]pl.east) -- node[right=2pt] {l.m.} ([shift={(1mm,-3.5mm)}]pl.east);
\end{tikzpicture}

		\caption*{\ref{itm:crossmatch2DFA-4} l.m.\tto l.m.}
	\end{subfigure}
	\hfill
	\begin{subfigure}{0.246\textwidth}
		\centering
		\begin{tikzpicture}[tapeseg/.append style={minimum width=2.5em}]
	\node[cell,thick] (sigma) {$\sigma$};
	\draw[thick,dashed] (sigma.north west) -- ++(-6mm,0) (sigma.south west) -- ++(-6mm,0);
	\draw[thick,dashed] (sigma.north east) -- ++(+6mm,0) (sigma.south east) -- ++(+6mm,0);
	\draw[dashed,shorten <=.1cm,thin] (sigma.south west) -- ++(0cm,-28mm);
	\draw[dashed,shorten <=.1cm,thin] (sigma.south east) -- ++(0cm,-28mm);

	\node[tapeseg,node distance=1pt] (leftalign) at ([xshift=-1.5mm]sigma) {};
	\node[tapeseg,node distance=1pt] (leftleftalign) at ([xshift=-7mm]sigma) {};
	\node[tapeseg,node distance=1pt] (rightalign) at ([xshift=1.5mm]sigma) {};
	\node[tapeseg,node distance=1pt] (rightrightalign) at ([xshift=7mm]sigma) {};

	\node (p1) [below=.5mm of rightalign] {$p_1$};
	\node (q1) [below=4mm of leftleftalign] {$q_1$};
	\draw[transition] ([shift={(5mm,2mm)}]p1.south) -- (p1.south);
	\draw[transition] (p1.south) -- (q1.south);

	\draw[dotted,thin] ([shift={(-6mm,-13mm)}]sigma.west) -- ([shift={(6mm,-13mm)}]sigma.east);

	\node (q2) [below=10.5mm of leftalign] {$q_2$};
	\draw[transition,smalldashed] (q1.south) .. controls +(-.5,-.3) .. (q2.south);
	\node (qdots) [below=-2 mm of q2] {$\vdots$};
	\node (qk) [below=-1mm of qdots] {$q_k$};
	\draw[transition] ([shift={(-5mm,2mm)}]qk.south) -- (qk.south);

	\node (p2) [below=11.5mm of rightrightalign] {$p_2$};
	\node (pdots) [below=-2mm of p2] {$\vdots$};
	\node (pl) [below=-1mm of pdots] {$p_l$};
	\draw[transition] ([shift={(-5mm,2mm)}]p2.south) -- (p2.south);
	\draw[transition] ([shift={(-5mm,2mm)}]pl.south) -- (pl.south);

	\scriptsize
	\draw[brace] ([shift={(.6mm,13.5mm)}]pl.east) -- node[right=2pt] {r.m.} ([shift={(.6mm,-3.5mm)}]pl.east);
\end{tikzpicture}

		\caption*{\ref{itm:crossmatch2DFA-5} r.m.\tto l.m.}
	\end{subfigure}
	\caption{Inductive definition of left and right matches. Each subfigure depicts $[q_1,\dots,q_k]$ \emph{left/right-matching} $[p_1,\dots,p_l]$. The part labeled r.m.\ (right-matching) and l.m.\ (left-matching) represents the inductive hypotheses, which is then extended with a new transition to make the new match.}
	\label{fig:2DFA-crossmatch}
\end{figure}

\subsubsection{From \TDFAs to \ONFAs}
We now prove that a \TDFA can be simulated by a \ONFA whose states are the valid crossing sequences of the \TDFA.
By \Cref{fact:crossing-2DFA-num} this also proves that an exponential number of states in the size of the \TDFA is sufficient for such a simulation.

Intuitively, the simulating machine puts together pieces of the computation of $A$ by guessing successive crossing sequences.
The full trajectory can be interpreted as a computation of the original machine, where acceptance is only possible if the right end-marker is passed while entering a final state.

Let $A:=(Q,\Sigma,\delta,q_I,F)$ be a \TDFA.
Define the \NFA $A':=(Q',\Sigma,\delta',I,F')$ where%
\footnote{ % TODO: move in models chapter
	Originally, the construction was introduced for \TDFAs without end-markers \cite{RabSco59}.
	In that case, $[q_I]$ is the crossing sequence of the left tape boundary, and it constitutes the initial state of the \NFA.
	Similarly, the right tape boundary is passed while the machine enters a final state; therefore, states storing crossing sequences $[q_f]$, for all $q_f\in F$, are set as final in the \NFA (for details see, \eg[,] \cite{HopUll79}).
	Simulating \TDFAs with end-markers is generally a stronger result, as they are in some cases exponentially smaller than equivalent \TDFAs without end-markers.
}
\begin{itemize}
	\item $Q'$ is the set of valid crossing sequences of $A$.
	\item $\delta'(c,\sigma):=\set{d \mid \text{$d$ is right-matched by $c$ with respect to~$\sigma$}}$.
\end{itemize}
Since \NFAs do not read end-markers, the starting state will be guessed among the crossing sequences that match $[q_I]$, which simulates entering the left end-marker from the left (as if $\lem$ was entered, rather than being the initial position).
For the same reason, final states correspond to crossing sequences that are compatible to the one passing the right end-marker:
\begin{itemize}
	\item $I:=\set{d\mid \text{$d$ is right-matched by $[q_I]$ with respect to $\lem$}}$.
	\item $F':=\set{d\mid \exists q_f\in F: \text{$d$ right-matches $[q_f]$ with respect to $\rem$}}$.
\end{itemize}

In order to prove that $A'$ accepts the same language as $A$, we prove two separate lemmas:
\begin{lemm}\label{lem:2DFAto1NFA-1}
	$\genlang(A)\subseteq\genlang(A')$.
\end{lemm}
\begin{proof}
	Given a word $w\in\genlang(A)$, consider the crossing sequences generated by an accepting computation of $A$ on $w$.

	$A$ begins its computation "entering" the left end-marker in state $q_I$, hence the crossing sequence of its left boundary is $[q_I]$.
	The crossing sequence of the next boundary, by construction, right-matches $[q_I]$ with respect to $\lem$.
	$A'$ guesses this crossing sequence by nondeterministically selecting its initial state.

	By construction, in an accepting computation each crossing sequence right-matches the one at the next boundary.
	Therefore $A'$ can guess, for each symbol, the correct matching crossing sequence, up to the one that corresponds to the left boundary of $\rem$ (after reading the last symbol of $w$).

	Because $A$ passes the right end-marker in a final state $q_f\in F$, the crossing sequence of the right boundary of $\rem$ is $[q_f]$.
	Since $A'$ guesses the correct crossing sequence at the left boundary of $\rem$ in the accepting computation of $A$, such a sequence must right-match $[q_f]$.
	Therefore, the current state of $A'$ is final and the machine accepts.
\end{proof}


\begin{lemm}\label{lem:2DFAto1NFA-2}
	$\genlang(A')\subseteq\genlang(A)$.
\end{lemm}
\begin{proof}
	Given a word $w:=w_1\cdots w_n\in\genlang(A')$, let $w_0:=\lem$, $w_{n+1}:=\rem$, and, for each $i\in\set{1,\dots,n}$, let $c_i$ be the crossing sequence (state) guessed by $A'$ after reading $w_i$ in a fixed accepting computation, with $c_0$ the guessed starting state.
	By definition of $\delta'$, for each $i\in\set{0,\dots,n-1}$, $c_i$ right-matches $c_{i+1}$ with respect to $w_{i+1}$.
	Furthermore let $c_{n+1}$ be a crossing sequence $[q_f]$, with $q_f\in F$, right-matched by $c_n$ with respect to $\rem$ (such sequence exists by definition of $F'$).
	We prove by induction on $i\in\set{0,\dots,n+1}$ that $c_i=[q_1,\dots,q_k]$ implies that
	\begin{statements}
		\item \label{lem:2DFAto1NFA-2-1} Given $w$ as input, $A$ first moves right from position $i$ into state $q_1$.
	\end{statements}
	Since $c_{n+1}=[q_f]$, with $q_f\in F$, by \Cref{lem:2DFAto1NFA-2-1} $A$ first passes $\rem$, in position $n+1$, in state $q_f$. Therefore $A$ accepts $w$.
	In order to show that the states chosen by $A'$ constitute in fact the crossing sequences of the accepting computation of $A$, we also show that $c_i=[q_1,\dots,q_k]$ implies that
	\begin{statements}[resume]
		\item \label{lem:2DFAto1NFA-2-2} For $j\in\set{1,\dots,\floor{\frac{k}{2}}}$, if $A$ is started in position $i$ in state $q_{2j}$, it eventually moves right from position $i$ in state $q_{2j+1}$.
	\end{statements}
	\begin{description}
		\item[Base] By definition of $I$, $c_0$ is right-matched by $[q_I]$ with respect to $\lem$. Since position $0$ contains $\lem$, which cannot be passed to the left, the right-match must follow from \Cref{itm:crossmatch2DFA-3}, hence $\delta(q_I,\lem)=(q_1,\tr)$ and $[~]$ left-matches $[q_2,\dots,q_k]$.
		      \Cref{lem:2DFAto1NFA-2-1} is satisfied as $A$ moves for the first time in position $1$ in state $q_1$.
		      \Cref{lem:2DFAto1NFA-2-2} follows from the fact that the left-match with $[~]$ must follow from repeated applications of \Cref{itm:crossmatch2DFA-4}, giving $\delta(q_{2j},\lem)=(q_{2j+1},\tr)$ for all values of $j$.
		\item[Step] Assume $A'$ transitions from state $c_{i-1}=[q_1,\dots,q_k]$ to state $c_i=[p_1,\dots,p_l]$ after reading $w_i$.
		      Since $c_{i-1}$ right-matches $c_i$, and $l$ and $k$ are odd ($c_{i-1}$ and $c_i$ being \emph{valid}), in the recursively defined right-match there must be an application of \Cref{itm:crossmatch2DFA-3} (after applying \Cref{itm:crossmatch2DFA-2} a number of times).
		      Therefore there exists a smallest $j_1$ such that $\delta(q_{j_1},w_i)=(p_1,\tr)$ (which proves \Cref{lem:2DFAto1NFA-2-1}) and that $[q_{j_1+1},\dots,q_k]$ left-matches $[p_2,\dots,p_l]$.
		      If the left-match follows from sole applications of \Cref{itm:crossmatch2DFA-4}, then \Cref{lem:2DFAto1NFA-2-2} is satisfied.
		      Otherwise by \Cref{itm:crossmatch2DFA-5} a right-match is derived and the argument repeats. \qedhere
	\end{description}
\end{proof}

Since the inclusion between $\genlang(A)$ and $\genlang(A')$ holds both ways, we can conclude that the two machines accept the same language.
In general:
\begin{thrm}\label{thm:2DFAto1NFA}
	Every $n$-state \TDFA can be simulated by a \ONFA with an exponential number of states in $n$.
\end{thrm}
\begin{proof}
	Let $A$ be an $n$-state \TDFA and $A'$ the \ONFA built with the construction presented above.
	By the conjunction of \Cref{lem:2DFAto1NFA-1,lem:2DFAto1NFA-2}, it holds that $\genlang(A)=\genlang(A')$, \ie $A'$ correctly simulates $A$.

	Furthermore, the number of states in $A'$ is the number of valid crossing sequences of $A$, which by \Cref{fact:crossing-2DFA-num} is exponential in $n$.
\end{proof}

% TODO: notes on the possibility to change the number of starting states to one without affecting the simulation cost (proof here or in appendix)


% TODO
\subsection{Sweeping \TDFAs}
% TODO: specify that in the case of sweeping automata (of any kind), rules ii and iv only apply to end-markers
\dots


\section{Transition tables}\label{sec:transtab2DFA}

\begin{defn}
	Given a \TDFA $A=(Q,\Sigma,\delta,q_I,F)$ a transition table of $A$ is a partial function $\tau:Q\to Q$.
	A transition table is assigned to each word. In particular:
	\begin{rules}
		\item for each $r\in Q$, $\tau_\lem(r):=s$, where $\delta(r,\lem)=(s,\tr)$,
		\item for each $r\in Q,w\in\Sigma\star,\sigma\in\Sigma$, $\tau_{w\sigma}(r):=s$, if either $\delta(r,a)=(s,\tr)$, or there exist an integer $k\ge1$ and states $q_1,\dots,q_k,r_1,\dots,r_k$ such that
		\begin{itemize}
			\item $\delta(r,a)=(q_1,\tl)$,
			\item $\tau_w(q_i)=r_i$ for all $i\in\set{1,\dots,k}$,
			\item $\delta(r_i,\sigma)=(q_{i+1},\tl)$ for all $i\in\set{1,\dots,k-1}$, and
			\item $\delta(q_k,\sigma)=(s,\tr)$.
		\end{itemize}
	\end{rules}
	We denote with $\transtabset$ the set of the transition tables of $A$.
	Notice that if $\tau_w$ is known, knowledge of $w$ is not needed to calculate $\tau_{w\sigma}$.
	For this reason we define, for each $\sigma\in\Sigma$, the function $t_\sigma:\transtabset\to\transtabset$ that maps a transition table $\tau_w$ to $\tau_{w\sigma}$.
\end{defn}


% TODO: make A' recognise lem w rem ? In that case, change q'_I to a new initial state and F' to the obvious

Let $A:=(Q,\Sigma,\delta,q_I,F)$ be a \TDFA.
Define the \DFA $A':=(Q',\Sigma,\delta',q'_I,F')$ where
\begin{itemize}
	\item $Q':=\transtabset\times Q$, where $\transtabset$ is the set of transition tables of $A$,
	\item $\delta'((\tau,r),\sigma):=(\pi,s)$, with $\pi:=t_\sigma(\tau)$ and $s:=\pi(r)$,
	\item $q'_I:=(\tau_\lem,s)$, where $\delta(q_I,\lem)=(s,\tr)$, and
	\item $F':=\set{(\tau,p)\mid t_\rem(\tau)(p)\in F}$.
\end{itemize}

A simple induction gives
\begin{lemm}
	For all~$w \in \Sigma^*$, the state reached by~$A'$ after reading~$w$ is $\delta(q'_I,w) = (\tau,p)$, for some $p\in Q$, if and only if~$\tau = \tau_w$.
\end{lemm}

\begin{lemm}\label{lem:transtab2DFA}
	For each $w\in\Sigma\star$ and $p\in Q$, $\delta'(q'_I,w)=(\tau_w,p)$ if and only if there exists a computation of $A$ on input $w$ that passes the last symbol of $w$ to the right for the first time in state $p$.
\end{lemm}
\begin{proof}
	By induction on $w$:
	\begin{description}
		\item[Base] $\delta'(q'_I,\emptyword)=q'_I=(\tau_\lem,s)$, where $\delta(q_I,\lem)=(s,\tr)$. The statement is trivially proven.
		      % TODO: PICTURE(s)
		\item[Step] Given $x\in\Sigma\star$ and $r\in Q$, assume $\delta'(q'_I,x)=(\tau_x,r)$ if and only if there exists a computation of $A$ on input $x$ that passes the last symbol of $x$ for the first time in state $r$, and let the tape contain $w=x\sigma$, with $\sigma\in\Sigma$.
		      \begin{description}
			      \item[$\Rightarrow$)] Assume $\delta'(q'_I,w)=(\tau_w,p)$.
			            We have $\delta'(q'_I,w)=\delta'(\delta'(q'_I,x),\sigma)$, with $\delta'(q'_I,x)=(\tau_x,r)$ for some $r\in Q$.
			            By induction hypothesis, there exists a computation of $A$ on input $x\sigma$ that reads $\sigma$ for the first time in state $r$.
			            By definition of $\delta'$, we have that $p=\tau_w(r)$.
			            By definition of transition table, there is a computation path that starts in state $r$ scanning $\sigma$ and first passes it to the right in state $p$ (after possibly scanning the left part of the tape that ends in $\sigma$).
			            By combining the two computations we obtain the desideratum.
			      \item[$\Leftarrow$)] Assume that there exists a computation of $A$ on input $w=x\sigma$ that passes $\sigma$ for the first time in state $r$.
			            Such computation can be split into two parts, the former ending when $\sigma$ is first read in a state $r$.
			            By applying the induction hypothesis on the first part of the computation, we obtain $\delta'(q'_I,x)=(\tau_x,r)$.
			            The second part of the computation describes a path that starts in state $r$ on $\sigma$ and first ends to its right in state $p$, therefore $\tau_w(r)=p$.
			            By definition of $\delta'$ we have $\delta'((\tau_x,r),\sigma)=(\tau_w,\tau_w(r))=(\tau_w,p)$.
		      \end{description}
	\end{description}
\end{proof}

We can now prove that the two machines are equivalent:
\begin{thrm}\label{thm:transtab2DFA}
	$\genlang(A')=\genlang(A)$.
\end{thrm}
\begin{proof}
	By \Cref{lem:transtab2DFA}, $\delta'(q'I_,w)=(\tau_w,p)$ if and only if there exists a computation of $A$ on input $w$ that reaches $\rem$ for the first time in state $p$.
	\begin{description}
		\item[$\genlang(A')\subseteq\genlang(A)$:] If $\delta'(q'I_,w)=(\tau_w,p)$ and $(\tau_w,p)$ is final, by definition $t_\rem(\tau_w)(p)\in F$, therefore $A$ on input $w$ ends its computation by passing $\rem$ in a final state.
		\item[$\genlang(A')\subseteq\genlang(A)$:] If $A$ accepts $w$ the accepting computation can be split as in the previous proof, giving $\delta'(q'I_,w)=(\tau_w,p)$ and $t_\rem(\tau_w)(p)\in F$, thus proving that $A'$ accepts.
	\end{description}
\end{proof}

Thanks to \Cref{thm:transtab2DFA} we can now show that
\begin{thrm}
	Every $n$-state \TDFA can be simulated by a \ODFA with an exponential number of states in $n$.
\end{thrm}
Which doubles as a proof of \Cref{thm:2DFAto1NFA}.
