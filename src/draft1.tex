\chapter{On sweeping \texorpdfstring{$k$-limited}{k-limited} automata}



% \section{Introduction}



\section{Crossing sequences}
Given a two-way machine and one of its possible computations over a certain input, consider a tape cell and one of the two boundaries at its sides.
A crossing sequence $[q_1,q_2,\dots]$ is the sequence of states that the machine enters while the head crosses such boundary, i.e.\ when entering and leaving the cell from/to the chosen side.

In general, because of loops and nondeterminism, a crossing sequence may be of infinite length.
However, we will see that we may restrict our interest to some finite set of finite sequences, depending on the model in question.

The first state of a crossing sequence is always entered by the machine while moving right, while subsequent crossings must be in alternate directions.
Therefore, odd-numbered elements of a crossing sequence represent states entered after surpassing the boundary to the right, while even-numbered elements represent states entered after surpassing the boundary to the left:
\begin{fact}\label{fact:crossing-parity}
	In a crossing sequence, odd-numbered states are entered after moves to the right, while even-numbered states are entered after moves to the left.
\end{fact}

Furthermore, because in the models we will consider the input is accepted by crossing the right boundary of the tape, the last crossing of any boundary must be towards the right:
\begin{fact}\label{fact:crossing-length}
	A crossing sequence of an accepting computation has odd length.
\end{fact}


\subsection{2DFAs}
We will now consider the specific case of two-way deterministic finite automata (2DFAs).
If a 2DFA accepts an input, the (one and only) computation on that input will never enter the same state twice while moving on a same cell in the same direction, otherwise the machine, being deterministic, would run in a loop and thus not accept.
In terms of crossing sequences:
\begin{fact}\label{fact:crossing-2dfa-parity}
	No crossing sequence of an accepting computation of a 2DFA may contain the same state in two odd- or even-numbered positions.
\end{fact}

We are now ready to give the definition of \emph{valid crossing sequence} for 2DFAs:
\begin{defn}
	Given a 2DFA, crossing sequence is called valid if and only if:
	\begin{itemize}
		\item its length is odd and bounded by $2n-1$, where $n$ is the number of states;
		\item no state is repeated in two odd- or even-numbered positions.
	\end{itemize}
\end{defn}

% TODO: note about the finiteness of the set of valid crossing sequences and explanation of its cardinality

Notice that, following facts \ref{fact:crossing-parity}, \ref{fact:crossing-length}, and \ref{fact:crossing-2dfa-parity}, the following holds:
\begin{fact}
	All crossing sequences of accepting computations of a 2DFA are valid.
\end{fact}

\subsubsection{Matching crossing sequences}
In an accepting computation, the two crossing sequences adjacent to each cell are compatible, meaning their states are the result of valid transitions given the cell's content.
In general, given a tape cell and two crossing sequences $[q_1,\dots,q_k]$ and $[p_1,\dots,p_l]$, we can check for local compatibility as follows.
If the head moves left from the cell in state $q_i$, restart the automaton on the cell in state $q_{i+1}$; if it moves right in state $p_i$, assume the automaton going back to the cell in state $p_{i+1}$.
Then check that the resulting computation is composed by states in the two crossing sequences in their correct order.

We recursively define left- and right-matching pairs of crossing sequences with respect to a cell containing symbol $\sigma$.
Sequence $[q_1,\dots,q_k]$ right-matches sequence $[p_1,\dots,p_l]$ when they are locally compatible starting with a transition to the right on the cell containing $\sigma$ in state $q_1$ and ending to its right, while we have a left-match when the initial transition is to the left in state $p_1$ (while still ending to the right of the cell).

% TODO: add pictures
\begin{defn}
	Given a 2DFA $A=(Q,\Sigma,\delta,q_0,F)$ we define left- and right-matching pairs of crossing sequences with respect to a symbol $\sigma\in\Sigma$ as follows:
	\begin{enumerate}[i.]
		\item The empty sequence $[~]$ right-matches and left-matches itself with respect to $\sigma$.
		\item \label{itm:crossmatch2dfa-2} If $[q_3,\dots,q_k]$ right-matches $[p_1,\dots,p_l]$ with respect to $\sigma$ and $\delta(q_1,\sigma)=(q_2,\tl)$, then $[q_1,\dots,q_k]$ right-matches $[p_1,\dots,p_l]$ with respect to $\sigma$. The computation enters the cell from the left in state $q_1$, turns around right away leaving it to the left in $q_2$, and eventually comes back to it in $q_3$, to which we apply induction.
		\item \label{itm:crossmatch2dfa-3} If $[q_2,\dots,q_k]$ left-matches $[p_2,\dots,p_l]$ with respect to $\sigma$ and $\delta(q_1,\sigma)=(p_1,\tr)$, then $[q_1,\dots,q_k]$ right-matches $[p_1,\dots,p_l]$ with respect to $\sigma$. The computation enters the cell from the right in state $p_1$, immediately leaves it to the left in $q_1$, and eventually comes back to it in $q_2$, to which we apply induction.
		\item Similarly to \ref{itm:crossmatch2dfa-2}, if $[q_1,\dots,q_k]$ left-matches $[p_3,\dots,p_l]$ with respect to $\sigma$ and $\delta(p_1,\sigma)=(p_2,\tr)$, then $[q_1,\dots,q_k]$ left-matches $[p_1,\dots,p_l]$ with respect to $\sigma$.
		\item Similarly to \ref{itm:crossmatch2dfa-3}, if $[q_2,\dots,q_k]$ left-matches $[p_2,\dots,p_l]$ with respect to $\sigma$ and $\delta(p_1,\sigma)=(q_1,\tl)$, then $[q_1,\dots,q_k]$ left-matches $[p_1,\dots,p_l]$ with respect to $\sigma$.
	\end{enumerate}
\end{defn}
Note that two crossing sequence need not be valid in order to match (e.g. the empty sequence).


\subsubsection{From 2DFA to 1NFA}
We will now prove that a 2DFA can be simulated by a 1NFA whose states are the valid crossing sequences of the 2DFA.
Intuitively, the simulating machine puts together pieces of the computation of $A$ by guessing successive crossing sequences. The full trajectory can be interpreted as a computation of the original machine, where acceptance is only possible if the right boundary of the tape is surpassed on a final state.
% TODO: update to the precise number of states?
\begin{thrm}
	Let $A$ be an $n$-state 2DFA. $A$ can be simulated by a 1NFA with an exponential number of states in $n$.
\end{thrm}
\begin{proof}
	Let $A:=(Q,\Sigma,\delta,q_0,F)$. Define $A':=(Q',\Sigma,\delta',q_0',F')$ where
	\begin{itemize}
		\item $Q'$ is the set of valid crossing sequences for $A$.
		\item $q_0'$ is the crossing sequence $[q_0]$.
		\item $F':=\set{[q_f] \mid q_f\in F}$.
		\item $\delta'(c,\sigma):=\set{d \mid \text{$d$ is right-matched by $c$ on input $\sigma$}}$.
	\end{itemize}

	If a word $w$ is in $\genlang(A)$, consider the crossing sequences generated by an accepting computation of $A$ on $w$.
	The leftmost cell of the tape is entered on state $q_0$ and the rightmost cell is left on a final state $q_f$, therefore the crossing sequences of the two boundaries in question are the initial state $[q_0]$ of $A$ and its final state $[q_f]$.
	Furthermore, each crossing sequence right-matches the one at the next boundary, therefore $A'$ can guess the proper crossing sequences and accept.

	Conversely, consider a word $w$ which is in $\genlang(A')$.
\end{proof}
