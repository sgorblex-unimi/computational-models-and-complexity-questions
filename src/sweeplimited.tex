\chapter{Sweeping limited automata}



\section{Introduction}
Given a natural number $k$, a sweeping $k$-limited automaton is a sweeping machine that can only write on its first $k$ sweeps (see \Cref{def:kla,def:sweeping} for a formal description).
In this chapter we show that, for every $k$, deterministic sweeping $k$-LAs characterize regular languages, by proving that they are as powerful as \NFAs. We also study this equivalence from a descriptional complexity point of view.

In order to prove the equivalence, we describe a way to construct, given a sweeping $k$-DLA, an equivalent NFA.
This simulation requires the sweeping $k$-DLA to be converted in a specific normal form, as described in \Cref{sec:equiv-swep-dla}.
\Cref{sec:crossseqswdla} extends the concept of crossing sequence, required by the simulation, to our model.
Finally, \Cref{sec:swkdla-to-NFA} describes the simulation and proves its correctness.



\section{A normal form for sweeping \texorpdfstring{$k$}{k}-DLAs}\label{sec:equiv-swep-dla}
A possibly problematic characteristic of sweeping $k$-DLAs is that they may accept during the \emph{active} part of computation, that is, while writing is still possible.
Here we describe a way to convert a sweeping $k$-DLA in a normal form that isolates the active computation by forcing the number of sweeps to be more than $k$ on every accepting computation.
The way the lower bound on the number of sweeps is chosen also forces a read-only sweep to the left, which will be useful in the main simulation of this chapter (\Cref{sec:swkdla-to-NFA}).
\Cref{tab:kprimesweep} shows the exact number of sweeps on each accepting computation of a sweeping $k$-DLA in normal form.
In such a machine, we refer to the first $k'$ sweeps as the active computation.
\begin{thrm}\label{thm:equiv-swep-dla}
	Let $k$ be a natural number and $k':=2\floor{\frac{k}{2}}+1$ (namely the smallest odd number greater or equal to $k$).
	Every sweeping $k$-DLA with $n$ states can be converted into an equivalent sweeping $k$-DLA with $2n+4$ states and the same working alphabet that performs at least $k'+2$ sweeps on every accepting computation.
\end{thrm}
\begin{proof}
	\newcommand{\ql}{q_{1\mathtt{L}}}
	\newcommand{\qr}{q_{1\mathtt{R}}}
	\newcommand{\qqr}{q_{2\mathtt{R}}}

	Let $A:=(Q,\Sigma,\Gamma,\delta,q_I,F)$ be a sweeping $k$-DLA.
	Define the sweeping $k$-DLA $A':=(Q',\Sigma,\Gamma,\delta',q'_I,F')$, where
	\begin{itemize}
		\item $Q'=(Q\times\set{0,1})\cup\set{q_0,\ql,\qr,\qqr}$,
		\item $q'_I:=(q_I,0)$,
		\item $F':=(F\times\set{1})\cup\set{\qr}$.
	\end{itemize}

	The automaton $A'$ initially simulates $A$ using the first component of the states in $Q\times\set{0,1}$ for a direct simulation of the states of $A$, and the second component to remember whether a symbol from $\Gamma_k$ has ever been read along the computation (bit set to $1$), or not (bit set to $0$).
	The automaton proceeds like this until $A$ would accept.
	That is, for each $q\in Q$ and $\sigma\in\Gamma$ such that $\delta(q,\sigma)=(q',\sigma',d)$:
	\begin{align*}
		\delta'((q,0),\sigma) & :=\begin{cases}
			                          ((q',1),\sigma',d) \qquad & \text{if }\sigma\in\Gamma_k \\
			                          ((q',0),\sigma',d) \qquad & \text{otherwise,}
		                          \end{cases} \\[1ex]
		\delta'((q,0),\lem)   & :=((q',0),\lem,\tr) \text,                                \\
		\delta'((q,1),\sigma) & :=((q',1),\sigma',d) \text,                               \\
		\delta'((q,1),\lem)   & :=((q',1),\lem,\tr) \text,
	\end{align*}
	and if $\delta(q,\rem)=(q',\rem,\tl)$:
	\begin{align*}
		\delta'((q,0),\rem) & =((q',0),\rem,\tl) \text, \\
		\delta'((q,1),\rem) & =((q',1),\rem,\tl) \text.
	\end{align*}

	When a transition that would bring the simulated device to accept (by passing the right end-marker in a final state) is detected, different cases occur:
	\begin{enumerate}
		\item If a symbol in $\Gamma_k$ was never read (\ie the bit of the second state component is set to $0$), less than $k+1<k'+2$ sweeps have been performed, therefore $A'$ enters the special state $q_0$, which has the task to perform the remaining sweeps (see below for details).
		\item If a symbol in $\Gamma_k$ has been read (\ie the stored bit is set to $1$), the automaton has performed at least $k+1$ sweeps. In particular:
		      \begin{enumerate}
			      \item If $k$ is even, $k'+2=k+3$, therefore in the worst case two more sweeps need to be performed.
			            This is done by entering the state $\ql$ and performing a sweep before turning right in state $\qr$ as $\lem$ is reached and continuing to the right until $\rem$ is passed, accepting.
			      \item If $k$ is odd, then $k'+2=k+2$. However, the $(k+1)$-th sweep was performed towards the left, thus, since the right end-marker has been reached, at least $k+2=k'+2$ sweeps have been performed. Therefore $A'$ can pass $\rem$ and accept.
		      \end{enumerate}
	\end{enumerate}

	Formally, if $\delta(q,\rem)=(q_f,\rem,\tr)$ (with $q_f$ final):
	\begin{align*}
		\delta'((q,0),\rem) & := (q_0,\rem,\tl) \text,                              \\
		\delta'((q,1),\rem) & = \begin{cases}
			                        (\ql,\rem,\tl) \qquad     & \text{if $k$ is even}   \\
			                        ((q_f,1),\rem,\tr) \qquad & \text{otherwise} \text.
		                        \end{cases}
	\end{align*}
	As for $\ql$ and $\qr$, for any $\sigma\in\Gamma$:
	\begin{align*}
		\delta'(\ql,\sigma) & :=(\ql,\sigma,\tl) \text, \\
		\delta'(\ql,\lem)   & :=(\qr,\lem,\tr) \text,   \\
		\delta'(\qr,\sigma) & :=(\qr,\sigma,\tr) \text, \\
		\delta'(\qr,\rem)   & :=(\qr,\rem,\tr) \text.
	\end{align*}

	The computation on the state $q_0$ proceeds by performing the remaining sweeps, writing for the $j$-th sweep an arbitrary symbol from $\Gamma_j$.
	Only one state is necessary for both directions, as the move will be to the right if the read symbol belongs to an even-indexed $\Gamma_i$ (or the left end-marker is read), and to the left otherwise.
	When a symbol from $\Gamma_k$ is read, similarly to above, the $k+1$ sweep is being performed. Two cases can occur:
	\begin{enumerate}
		\item If $k$ is even, the automaton is moving to the right, and must finish the current sweep and perform two more. This is done using a state $\qqr$ (first sweep) and the states $\ql$ and $\qr$ (last two sweeps).
		\item If $k$ is odd, the automaton is moving to the left, and must finish the current sweep before performing a last one to the right and accept. As before, this can be done using the states $\ql$ and $\qr$.
	\end{enumerate}

	Formally, given $\sigma\in\Gamma_i$, $i\ne k$, and called $\sigma_j$ an arbitrary element of $\Gamma_j$:
	\begin{align*}
		\delta'(q_0,\sigma) & = \begin{cases}
			                        (q_0,\sigma_{i+1},\tr) \qquad & \text{if $i$ is even}   \\
			                        (q_0,\sigma_{i+1},\tl) \qquad & \text{otherwise} \text,
		                        \end{cases} \\[1ex]
		\delta'(q_0,\lem)   & = (q_0,\lem,\tr) \text,                                   \\
		\delta'(q_0,\rem)   & = (q_0,\rem,\tl) \text,
	\end{align*}
	and given $\sigma\in\Gamma_k$:
	\begin{align*}
		\delta'(q_0,\sigma) & = \begin{cases}
			                        (\qqr,\sigma,\tr) \qquad & \text{if $k$ is even}   \\
			                        (\ql,\sigma,\tl) \qquad  & \text{otherwise} \text.
		                        \end{cases}
	\end{align*}
	As for $\qqr$, for any $\sigma\in\Gamma$:
	\begin{align*}
		\delta'(\qqr,\sigma) & :=(\qqr,\sigma,\tr) \text,       \\
		\delta'(\qqr,\rem)   & :=(\ql,\rem,\tl) \text. \qedhere
	\end{align*}
\end{proof}

\begin{table}
	\centering
	\begin{tabular}{lcc}
		\toprule
		~        & $k'$  & \# sweeps \\
		\midrule
		$k$ even & $k+1$ & $k+3$     \\
		$k$ odd  & $k$   & $k+2$     \\
		\bottomrule
	\end{tabular}
	\caption{Summary of the values of $k'$ and the number of sweeps performed by the output automaton from the construction in \Cref{thm:equiv-swep-dla}.}
	\label{tab:kprimesweep}
\end{table}



\section{Crossing sequences}\label{sec:crossseqswdla}


\subsection{Matching crossing sequences}
% It can technically hold for any LBA
\begin{defn}
	Given a $k$-DLA $A:=(Q,\Sigma,\Gamma,\delta,q_I,F)$, we define left- and right-matching pairs of crossing sequences with respect to a symbol $\sigma\in\Gamma$ as follows:
	\begin{rules}
		\item \label{itm:crossmatchswepDLA-1} The empty sequence $[~]$ right-matches and left-matches itself with respect to~$\sigma$.
		\item \label{itm:crossmatchswepDLA-2} If $[q_3,\dots,q_h]$ right-matches $[p_1,\dots,p_l]$ with respect to~$\sigma'$ and $\delta(q_1,\sigma)=(q_2,\sigma',\tl)$, then $[q_1,\dots,q_h]$ right-matches $[p_1,\dots,p_l]$ with respect to~$\sigma$.
		The computation enters the cell from the left in state $q_1$, turns around right away leaving it to the left in $q_2$ writing $\sigma'$, and eventually comes back to it in $q_3$, to which we apply induction.
		\item \label{itm:crossmatchswepDLA-3} If $[q_2,\dots,q_h]$ left-matches $[p_2,\dots,p_l]$ with respect to~$\sigma'$ and $\delta(q_1,\sigma)=(p_1,\sigma',\tr)$, then $[q_1,\dots,q_h]$ right-matches $[p_1,\dots,p_l]$ with respect to~$\sigma$.
		The computation enters the cell from the right in state $p_1$, immediately leaves it to the left in $q_1$ writing $\sigma'$, and eventually comes back to it in $q_2$, to which we apply induction.
		\item \label{itm:crossmatchswepDLA-4} Similarly to \Cref{itm:crossmatchswepDLA-2}, if $[q_1,\dots,q_h]$ left-matches $[p_3,\dots,p_l]$ with respect to~$\sigma'$ and $\delta(p_1,\sigma)=(p_2,\sigma',\tr)$, then $[q_1,\dots,q_h]$ left-matches $[p_1,\dots,p_l]$ with respect to~$\sigma$.
		\item \label{itm:crossmatchswepDLA-5} Similarly to \Cref{itm:crossmatchswepDLA-3}, if $[q_2,\dots,q_h]$ right-matches $[p_2,\dots,p_l]$ with respect to~$\sigma'$ and $\delta(p_1,\sigma)=(q_1,\sigma',\tl)$, then $[q_1,\dots,q_h]$ left-matches $[p_1,\dots,p_l]$ with respect to~$\sigma$.
	\end{rules}
\end{defn}
% TODO: specify that in the case of sweeping automata (of any kind), rules ii and iv only apply to end-markers

% As for previous, it can technically hold for any LBA
\begin{defn}
	Given a $k$-DLA, we define the partial function $\last:\crosseqset\times\crosseqset\times\Gamma\to\Gamma$ as follows:

	Given crossing sequences $c=[q_1,\dots,q_h]$ and $d=[p_1,\dots,p_l]$ and a symbol $\sigma\in\Gamma$:
	\begin{itemize}
		\item If $c$ right-matches $d$ with respect to $\sigma$ via \Cref{itm:crossmatchswepDLA-2}, then $\delta(q_1,\sigma)=(q_2,\sigma',\tl)$ for some $\sigma'$, and we define:
		      \begin{equation*}
			      \last(c,d,\sigma):=\begin{cases}
				      \sigma'                                        & \text{if } c=[q_1,q_2] \text{ and } d=[~], \\
				      \last([q_3,\dots,q_h],[p_1,\dots,p_l],\sigma') & \text{otherwise}.
				      % \last([q_3,\dots,q_h],d,\sigma') & \text{otherwise}.
			      \end{cases}
		      \end{equation*}
		\item If $c$ right-matches $d$ with respect to $\sigma$ via \Cref{itm:crossmatchswepDLA-3}, then $\delta(q_1,\sigma)=(p_1,\sigma',\tr)$ for some $\sigma'$, and we define:
		      \begin{equation*}
			      \last(c,d,\sigma):=\begin{cases}
				      \sigma'                                        & \text{if } c=[q_1] \text{ and } d=[p_1],~~ \\
				      \last([q_2,\dots,q_h],[p_2,\dots,p_l],\sigma') & \text{otherwise}.
			      \end{cases}
		      \end{equation*}
		\item If $c$ left-matches $d$ with respect to $\sigma$ via \Cref{itm:crossmatchswepDLA-4}, then $\delta(p_1,\sigma)=(p_2,\sigma',\tr)$ for some $\sigma'$, and we define:
		      \begin{equation*}
			      \last(c,d,\sigma):=\begin{cases}
				      \sigma'                                        & \text{if } c=[~] \text{ and } d=[p_1,p_2], \\
				      \last([q_1,\dots,q_h],[p_3,\dots,p_l],\sigma') & \text{otherwise}.
				      % \last(c,[p_3,\dots,p_l],\sigma') & \text{otherwise}.
			      \end{cases}
		      \end{equation*}
		\item If $c$ left-matches $d$ with respect to $\sigma$ via \Cref{itm:crossmatchswepDLA-5}, then $\delta(p_1,\sigma)=(q_1,\sigma',\tl)$ for some $\sigma'$, and we define:
		      \begin{equation*}
			      \last(c,d,\sigma):=\begin{cases}
				      \sigma'                                        & \text{if } c=[q_1] \text{ and } d=[p_1],~~ \\
				      \last([q_2,\dots,q_h],[p_2,\dots,p_l],\sigma') & \text{otherwise}.
			      \end{cases}
		      \end{equation*}
	\end{itemize}
\end{defn}


\subsection{Active crossing sequences}
\begin{defn}
	Given a sweeping $k$-LA $A$, a crossing sequence over the states of $A$ is called \emph{active} if and only if its length is exactly $k':=2\floor{\frac{k}{2}}+1$.
\end{defn}



\section{Simulation by NFA}\label{sec:swkdla-to-NFA}
We are now ready to describe the simulation of sweeping $k$-DLAs by 1\NFAs.
The simulation combines three techniques: crossing sequences, the automaton for the reversal, and transition tables (see \Cref{sec:transtab2DFA}).

We want to build the simulation so that there is a one-to-one correspondence between accepting computations of the sweeping $k$-DLA in normal form and the ones of the simulating NFA.
To do so we divide each of the former in three parts:
\begin{itemize}
	\item the first part is composed by the first $k'$ sweeps, ending to the right with the tape in a frozen state. We simulate this part by using crossing sequences, which also give us a way to calculate the last symbol written on each cell,
	\item the second part consists in the following sweep to the left, simulated by the reversal construction,
	\item the third part corresponds to the remaining sweeps and it's simulated by transition tables.
\end{itemize}

Let $A:=(Q,\Sigma,\Gamma,\delta,q_I,F)$ be a sweeping $k$-DLA.
Let $\crosseqset$ be the set of its active crossing sequences and $\transtabset$ is the set of its transition tables.
Without loss of generality (\Cref{thm:equiv-swep-dla}), $A$ performs at least $k'+2$ sweeps on each accepting computation.

Define the NFA $A':=(Q',\Sigma,\delta',I,F')$, where
\begin{itemize}
	\item $Q'=\crosseqset\times Q\times\transtabset\times Q$,
	\item $\delta'((c,p,\tau,r),\sigma):=\{(d,q,\pi,s) \mid \text{$d$ is right-matched by $c$ w.r.t. $\sigma$,}$\\
	      \newcommand{\phant}{\phantom{\delta'((c,p,\tau,r),\sigma):=\{(d,q,\pi,s) \mid ~}}
	      $\phant \sigma':=\last(c,d,\sigma)$, \\
	      $\phant \delta(q,\sigma')=(p,\sigma',\tl)$, \\
	      $\phant \pi=\trapp_{\sigma'}(\tau)$, \\
	      $\phant \crossphantom{s=\pi(r)}{\text{$d$ is right-matched by $c$ w.r.t. $\sigma$,}}\}$,
	\item $I:=\{(c,p,\tau_\lem,s) \mid \text{$c$ is right-matched by $[q_I]$ w.r.t. $\lem$},\\
		      \phantom{I:=\{(c,p,\tau_\lem,s) \mid~} \crossphantom{\delta(p,\lem)=(s,\lem,\tr)}{\text{$c$ is right-matched by $[q_I]$ w.r.t. $\lem$},}\}$, and
	\item $F':=\{(c,p,\tau,r) \mid c=[q_1,\dots,q_{k'}]$, \\
	      \renewcommand{\phant}{\phantom{F':=\{(c,p,\tau,r) \mid ~}}
	      $\phant \delta(q_{k'},\rem)=(p,\rem,\tr)$, \\
	      $\phant \crossphantom{t_\rem(\tau)(r)\in F}{\delta(q_{k'},\rem)=(p,\rem,\tr)\text,}\}$.
\end{itemize}

\begin{lemm}\label{lem:swkLAtoNFA-1}
	$\genlang(A)\subseteq\genlang(A')$.
\end{lemm}

\begin{lemm}\label{lem:swkLAtoNFA-2}
	$\genlang(A')\subseteq\genlang(A)$.
\end{lemm}

\begin{thrm}\label{thm:swkLAtoNFA}
	$\genlang(A')=\genlang(A)$.
\end{thrm}

Thanks to \Cref{thm:swkLAtoNFA} we can now show that
\begin{thrm}
	Every $n$-state sweeping $k$-LA can be simulated by a 1NFA with an exponential number of states in $n$.
\end{thrm}
