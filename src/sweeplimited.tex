\chapter{Sweeping limited automata}



\section{Crossing sequences}


\subsection{Matching crossing sequences}
% It can technically hold for any LBA
\begin{defn}
	Given a $k$-DLA $A:=(Q,\Sigma,\Gamma,\delta,q_I,F)$, we define left- and right-matching pairs of crossing sequences with respect to a symbol $\sigma\in\Gamma$ as follows:
	\begin{rules}
		\item \label{itm:crossmatchswepDLA-1} The empty sequence $[~]$ right-matches and left-matches itself with respect to~$\sigma$.
		\item \label{itm:crossmatchswepDLA-2} If $[q_3,\dots,q_h]$ right-matches $[p_1,\dots,p_l]$ with respect to~$\sigma'$ and $\delta(q_1,\sigma)=(q_2,\sigma',\tl)$, then $[q_1,\dots,q_h]$ right-matches $[p_1,\dots,p_l]$ with respect to~$\sigma$.
		The computation enters the cell from the left in state $q_1$, turns around right away leaving it to the left in $q_2$ writing $\sigma$, and eventually comes back to it in $q_3$, to which we apply induction.
		\item \label{itm:crossmatchswepDLA-3} If $[q_2,\dots,q_h]$ left-matches $[p_2,\dots,p_l]$ with respect to~$\sigma'$ and $\delta(q_1,\sigma)=(p_1,\sigma',\tr)$, then $[q_1,\dots,q_h]$ right-matches $[p_1,\dots,p_l]$ with respect to~$\sigma$.
		The computation enters the cell from the right in state $p_1$, immediately leaves it to the left in $q_1$ writing $\sigma$, and eventually comes back to it in $q_2$, to which we apply induction.
		\item \label{itm:crossmatchswepDLA-4} Similarly to \Cref{itm:crossmatchswepDLA-2}, if $[q_1,\dots,q_h]$ left-matches $[p_3,\dots,p_l]$ with respect to~$\sigma'$ and $\delta(p_1,\sigma)=(p_2,\sigma',\tr)$, then $[q_1,\dots,q_h]$ left-matches $[p_1,\dots,p_l]$ with respect to~$\sigma$.
		\item \label{itm:crossmatchswepDLA-5} Similarly to \Cref{itm:crossmatchswepDLA-3}, if $[q_2,\dots,q_h]$ right-matches $[p_2,\dots,p_l]$ with respect to~$\sigma'$ and $\delta(p_1,\sigma)=(q_1,\sigma',\tl)$, then $[q_1,\dots,q_h]$ left-matches $[p_1,\dots,p_l]$ with respect to~$\sigma$.
	\end{rules}
\end{defn}

% As for previous, it can technically hold for any LBA
\begin{defn}
	Given a $k$-DLA, we define the partial function $\last:\crosseqset\times\crosseqset\times\Gamma\to\Gamma$ as follows:

	Given crossing sequences $c=[q_1,\dots,q_h]$ and $d=[p_1,\dots,p_l]$ and a symbol $\sigma\in\Gamma$:
	\begin{itemize}
		\item If $c$ right-matches $d$ with respect to $\sigma$ via \Cref{itm:crossmatchswepDLA-2}, then $\delta(q_1,\sigma)=(q_2,\sigma',\tl)$ for some $\sigma'$, and we define:
		      \begin{equation*}
			      \last(c,d,\sigma):=\begin{cases}
				      \sigma'                                        & \text{if } c=[q_1,q_2]\land d=[~], \\
				      \last([q_3,\dots,q_h],[p_1,\dots,p_l],\sigma') & \text{otherwise}.
				      % \last([q_3,\dots,q_h],d,\sigma') & \text{otherwise}.
			      \end{cases}
		      \end{equation*}
		\item If $c$ right-matches $d$ with respect to $\sigma$ via \Cref{itm:crossmatchswepDLA-3}, then $\delta(q_1,\sigma)=(p_1,\sigma',\tr)$ for some $\sigma'$, and we define:
		      \begin{equation*}
			      \last(c,d,\sigma):=\begin{cases}
				      \sigma'                                        & \text{if } c=[q_1]\land d=[p_1],~~ \\
				      \last([q_2,\dots,q_h],[p_2,\dots,p_l],\sigma') & \text{otherwise}.
			      \end{cases}
		      \end{equation*}
		\item If $c$ left-matches $d$ with respect to $\sigma$ via \Cref{itm:crossmatchswepDLA-4}, then $\delta(p_1,\sigma)=(p_2,\sigma',\tr)$ for some $\sigma'$, and we define:
		      \begin{equation*}
			      \last(c,d,\sigma):=\begin{cases}
				      \sigma'                                        & \text{if } c=[~]\land d=[p_1,p_2], \\
				      \last([q_1,\dots,q_h],[p_3,\dots,p_l],\sigma') & \text{otherwise}.
				      % \last(c,[p_3,\dots,p_l],\sigma') & \text{otherwise}.
			      \end{cases}
		      \end{equation*}
		\item If $c$ left-matches $d$ with respect to $\sigma$ via \Cref{itm:crossmatchswepDLA-5}, then $\delta(p_1,\sigma)=(q_1,\sigma',\tl)$ for some $\sigma'$, and we define:
		      \begin{equation*}
			      \last(c,d,\sigma):=\begin{cases}
				      \sigma'                                        & \text{if } c=[q_1]\land d=[p_1],~~ \\
				      \last([q_2,\dots,q_h],[p_2,\dots,p_l],\sigma') & \text{otherwise}.
			      \end{cases}
		      \end{equation*}
	\end{itemize}
\end{defn}


\subsection{Active crossing sequences}
% TODO: definition of "active" crossing sequence (fixed length ~k)
\dots


\section{Simulation by NFA}
Let $A:=(Q,\Sigma,\Gamma,\delta,q_I,F)$ a sweeping $k$-DLA.
Let $\crosseqset$ the set of its active crossing sequences, $\transtabset$ is the set of its transition tables, and $\RM\subseteq(\crosseqset\times\crosseqset\times\Gamma\times\Gamma)$ the relation of its right-matching crossing sequences.

Define the NFA $A':=(Q',\Sigma',\delta',I,F')$, where
\begin{itemize}
	\item $Q'=\mathcal C\times\mathcal T\times Q$,
	\item $\Sigma'=\Sigma\cup\set{\lem,\rem}$,
	\item $\delta'((c,\tau,r),\sigma):=\set{(d,\pi,s) \mid \text{$d$ is right-matched by $c$ w.r.t. $\sigma$ and,}\\
			      \text{\hspace{8em}defined $\sigma':=\last(c,d,\sigma)$,\quad$\pi=\trapp_{\sigma'}(\tau)$ and $s=\pi(r)$}}$,
	\item $I:=\set{(c,\tau_\emptyword,r) \mid \text{$c=[q_1,\dots,r]$ is right-matched by $[q_I]$ w.r.t. $\lem$}}$, and
	\item $F':=\set{(c,\tau,s)\mid s\in F}$.
\end{itemize}
