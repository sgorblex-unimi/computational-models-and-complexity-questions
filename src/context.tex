\chapter{Context and motivation}



\section{Formal language and automata theory}
Formal language theory is a fundamental field of theoretical computer science which studies languages, \ie sets of strings, and finite ways to represent them.

\paragraph{Language descriptions} These finite representations span different paradigms: for example, we might state a property that holds on the strings belonging to the language, or construct a generative or a recognizing model.
Generative models, typically grammars, define a list of rules that can be used to generate the strings in the desired language.
Recognizing computational models consist in abstract machines that can determine, in a certain number of steps, if a given string belongs to the chosen language.

\paragraph{The Chomsky hierarchy} Different models for the description of languages characterize possibly different language classes, originating what is commonly known as the Chomsky hierarchy \cite{Cho56}.
\ironic{Although his attempt to create a model generating the English language failed miserably, \citeauthor{Cho56} accidentally founded an entire field of computer science, which is somehow still researched to this day by people like the undersigned.}{}
The hierarchy, originally described for grammars, spans four language classes, each one strictly containing the following ones (see \Cref{tab:chomsky} for a summary):
\begin{itemize}
	\item \emph{recursively enumerable}, characterized by Turing machines (\TM),
	\item \emph{context-sensitive}, characterized by linear bounded automata (\LBA),
	\item \emph{context-free}, characterized by pushdown automata (\PDA),
	\item \emph{regular}, characterized by finite state automata (\DFA).
\end{itemize}

\begin{table}
	\centering
	\renewcommand{\arraystretch}{1.2}
	\begin{tabular}{lll}
		\toprule
		\textsc{Language class} & \textsc{Recognizer}          & \textsc{Production rules}          \\
		\midrule
		Recursively enumerable  & Turing machine               & $\gamma\to\alpha$                  \\
		Context-sensitive       & \makecell[lt]{Linear bounded                                      \\automata} & $\alpha A\beta\to\alpha\gamma\beta$   \\
		Context-free            & Pushdown automata            & $A\to\alpha$                       \\
		Regular                 & Finite state automata        & \makecell[lt]{$A\to a$, $A\to aB$, \\$A\to\emptyword$} \\
		\bottomrule
	\end{tabular}
	\caption{The Chomsky hierarchy. Each class is listed along its canonical recognizer and the type of rules in generating grammars. In the production rules, $\emptyword$ is the empty string, $a$ is a terminal symbol, $A$ and $B$ are non-terminal symbols, and $\alpha$, $\beta$ and $\gamma$ are generic strings, with $\gamma\ne\emptyword$.}
	\label{tab:chomsky}
\end{table}

\paragraph{Nondeterminism} A very important concept in formal language theory is the one of nondeterminism.
At a given time of its computation, a nondeterministic machine has possibly multiple choices on the next step, which implies the possibility of several computation paths given the same input.
We say that a nondeterministic machine accepts a string if there exists a computation path over that string which brings the machine to accept.

One of the very first problems of the field was to determine if the nondeterministic and deterministic versions of the same computational model accept the same language class.
While this was found to be the case for the canonical recognizers of recursively enumerable, context-sensitive and regular languages, \citeauthor{Fis63} and \citeauthor{Sch63} proved that deterministic pushdown automata accept a smaller class than their nondeterministic counterpart \cite{Fis63,Sch63}.
In particular, concerning the regular language class, \citeauthor{RabSco59} found in \citeyear{RabSco59} a construction, today known as the subset construction, that lets a deterministic finite automaton (\NFA) simulate a deterministic one (\DFA) \cite{RabSco59}.

\paragraph{Two-way models} In the same article, \citeauthor{RabSco59} defined two-way nondeterministic and deterministic finite automata (\TNFA and \TDFA).
These models act in a similar way to the classic, one-way finite automata (also indicated with \ONFA and \ODFA), where the input is read one symbol at a time and the machine changes state depending on the scanned symbol and the current state; however, while one-way machines read the input symbols subsequently from left to right, two-way models can move left or right freely, as described by the transition rules.
\citeauthor{RabSco59} proved that these new models also characterize regular languages, by simulating them with a \ONFA with a construction based on crossing sequences (more in \Cref{sec:crossseq2DFA}).
The same result was found in parallel by \citeauthor{She59}, who used a different construction based on transition tables (see \Cref{sec:transtab2DFA}) \cite{She59}.



\section{Descriptional complexity}

\paragraph{Model size bounds} With many equivalent computational models accepting the same language classes, questions naturally arose about comparisons between them.
In \citeyear{MeyFis71}, \citeauthor{MeyFis71} proposed to compare different descriptions for a language class by their size, that is, by how succinctly they can describe the chosen language \cite{MeyFis71}.
Concretely, this typically means comparing the size of the components of the various models, such as the number of states of the size of the working alphabet.

From the construction given by \cite{RabSco59}, it followed that each \NFA can be simulated by a \DFA by paying an exponential cost in the number of states.
In their article, \citeauthor{MeyFis71} proved that there exist languages for which this difference in size cannot be avoided, therefore giving the a tight bound on the cost of removing nondeterminism from one-way finite automata.

\paragraph{The Sakoda and Sipser conjecture} Given the exponential distance between the sizes of the nondeterministic and deterministic one-way models, the natural next step is to compare their two-way counterparts.
The afore-mentioned constructions by crossing sequences (\cite{RabSco59}) and transition tables (\cite{She59}) gave an exponential upper bound on the simulation of \TNFA or \TDFA by \ONFA or \ODFA.
An exponential lower bound on the simulation of \TNFA by \ODFA is a direct consequence of the one on the simulation of \ONFA by \ODFA.
Later, thanks to \citeauthor{Bir93} (\citeyear{Bir93}) and \citeauthor{Kap05} (\citeyear{Kap05}), matching exponential lower bounds were found for the simulations of \TDFA by \ONFA and \ODFA and of \TNFA by \ONFA \cite{Bir93,Kap05}.
These descriptional complexity results for regular language acceptors, summarized in \Cref{tab:sims-core-general-context}, leave only two open problems: the simulations by \TDFA of \TNFA and \ONFA.

\begin{table}
	\centering
	\renewcommand{\hstdef}{.35}
\begin{tabular}{|l|l|l|p{2.9em}|l|}
	\hline
	~     & \ODFA   & \ONFA     & \TDFA                          & \TNFA     \\ \hline
	\ODFA & \cY     & $\Cctriv$ & $\Cctriv$                      & $\Cctriv$ \\ \hline
	\ONFA & $\Cexp$ & \cY       & \cR $\le\Cexp\hst$ $\ge\Cpoly$ & $\Cctriv$ \\ \hline
	\TDFA & $\Cexp$ & $\Cexp$   & \cY                            & $\Cctriv$ \\ \hline
	\TNFA & $\Cexp$ & $\Cexp$   & \cR $\le\Cexp\hst$ $\ge\Cpoly$ & \cY       \\ \hline
\end{tabular}

	\caption{Costs of the simulations between regular language recognisers.}
	\label{tab:sims-core-general-context}
\end{table}

These simulation costs were first discussed by \cite{SakSip78} in \citeyear{SakSip78}, as the problem of using deterministic two-way movement to simulate nondeterministic computations.
In their article, they describe a parallelism between this problem and the famous \textsc{P} versus \textsc{NP} problem for time complexity classes, and conjecture that the costs of those simulations are exponential.

% TODO: implications on L vs NL
Later [\dots]

% TODO: see thesis abroad project about ways the saksip conj has been fought
Because of the importance of its repercussions, the Sakoda and Sipser problem has been tackled in many ways over the past decades, [\dots]
