\chapter{Unary once-marking \texorpdfstring{$1$}{1}-limited automata}
Unary languages are languages over a one-letter alphabet.
In formal language theory, they have been studied extensively, one of the main reasons being that, in the unary case, the classes of context-free and regular languages coincide \cite{GinRic62}.

The descriptional complexity of the unary case has been thoroughly investigated over the years.
Because of the equality of the two classes, many computational models are equivalent: typical regular language acceptors such as one- and two-way finite state automata as well as typical context-free models such as pushdown automata.
The relative succinctness of these models has been inspected by building simulations and studying lower bounds (see, \eg[,] \cite{GefMer+03,MerPig01,Pig08,PigSha+02}).
At the same time, many bounds have been proven on the descriptional complexity of operations on unary languages (see, \eg[,] \cite{HolKut03,KunOkh12,MerPig05,PigSha02}).
Another reason why the unary case is studied extensively is that proving the Sakoda and Sipser conjecture, even in the unary case, would prove the separation of the space complexity classes $\cL$ and $\cNL$ (see \Cref{sec:context-descrcomp}).

Once-marking $1$-limited automata consist in machines with a single tape, delimited by end-markers ($\lem$,$\rem$), that can only write a single time, on the first visit of a tape cell.
They have been introduced and studied by \citeauthor{PigPri23a}, who proved a tight, double exponential bound for the simulation by \ODFAs in the general case and a polynomial one by \TDFA in the deterministic case.

In this chapter, we revise the previous work on once-marking $1$-limited automata, proving a previously mistaken bound on the simulation by \ONFA (\Cref{sec:oncemarking-general}).
We then gather previous results in order to obtain a map of the costs in the unary case (\Cref{sec:oncemarking-unary}).
Finally, we discuss ideas to potentially improve these bounds, by applying to this model techniques form different variants of regular languages and their recognizers (\Cref{sec:oncemarking-ideas}).



\section{The general case}\label{sec:oncemarking-general}
Once-marking $1$-limited automata (\OMOLA) are a restriction of $1$-limited automata (see \Cref{def:kla}) that can only mark a single tape cell throughout the entire computation.

In the seminal paper, \citeauthor{PigPri23a} show that in the general case there is a double exponential tight bound for the simulation of a \OMOLA by a \ODFA \cite{PigPri23a}.
In particular, the upper bound is inherited from the simulation of \OLA, while a new lower bound is proven with a restricted case of \OMOLA that is also sweeping and can only use nondeterminism in the first visit to tape cells.
Regarding fully deterministic once-marking $1$-limited automata (\OMODLA), the authors build a construction based on the simulation of the machine's computation tree by a \TDFA, which only requires a cubic size increase.

\begin{table}
	\centering
	\renewcommand{\arraystretch}{1.1}
\renewcommand{\hstdef}{.55}
\begin{tabular}{|l|l|l|p{3.1em}|l|l|p{4.3em}|}
	\hline
	~       & \ODFA              & \ONFA     & \TDFA                                        & \TNFA              & \OMOLA    & \OMODLA                                       \\ \hline
	\ODFA   & \cY                & $\Cctriv$ & $\Cctriv$                                    & $\Cctriv$          & $\Cctriv$ & $\Cctriv$                                     \\ \hline
	\ONFA   & $\Cexp$            & \cY       & \cR $\le\Cexp\hst$ $\ge\Cpoly$               & $\Cctriv$          & $\Cctriv$ & \cB $\le\Cexp\hst[1.75]$ $\ge\Cpoly$          \\ \hline
	\TDFA   & $\Cexp$            & $\Cexp$   & \cY                                          & $\Cctriv$          & $\Cctriv$ & $\Cctriv$                                     \\ \hline
	\TNFA   & $\Cexp$            & $\Cexp$   & \cR $\le\Cexp\hst$ $\ge\Cpoly$               & \cY                & $\Cctriv$ & \cB $\le\Cexp\hst[1.75]$ $\ge\Cpoly$          \\ \hline
	\OMOLA  & \rbt[.2]{$\Cdexp$} & $\Cexp$   & \cG \rbt[.2]{$\le\Cdexp\hst[.1]$} $\ge\Cexp$ & $\Cexp$            & \cY       & \cG \rbt[.2]{$\le\Cdexp\hst[1.3]$} $\ge\Cexp$ \\ \hline
	\OMODLA & $\Cexp$            & $\Cexp$   & \rbt[.1]{$O(n^3)$}                           & \rbt[.1]{$O(n^3)$} & $\Cctriv$ & \cY                                           \\ \hline
\end{tabular}
%
	\caption{Costs of the simulations between once-marking $1$-limited automata and other regular language acceptors.}
	\label{tab:sims-om-general-oncemarking}
\end{table}

We summarize these and other related results in \Cref{tab:sims-om-general-oncemarking}.
We point out that the simulation of \OMODLA by \ONFA has been mistakenly labeled as polynomial in \cite{PigPri23a}, while there is indeed an exponential lower bound:
\begin{thrm}\label{thm:OM1DLAto1NFAlower}
	For each integer $n$ there exists an $n$-state \OMODLA $A$ such that every \ONFA recognizing $\genlang(A)$ requires an exponential number of states in $n$.
\end{thrm}
\begin{proof}
	Consider the language $\double:=\set{ww\mid w\in\set{a,b}^n}$.
	A \OMODLA can recognize $\double$ as follows:
	\begin{enumerate}
		\item By using an $n$-state counter, the automaton marks the $(n+1)$-th input cell (if there is no such cell, it rejects).
		\item The tape head is reset to the first input symbol by first reaching the left end-marker and then moving right once. The scanned symbol is saved in the finite state memory.
		\item\label{itm:double-step3} By moving $n$ times to the right, the head is brought to the first symbol in the second "half" of the input. The scanned symbol is compared to the one saved.
		      If they are different, the machine rejects.
		\item\label{itm:double-step4} By moving $n$ times to the left followed by one to the right (or simply $n-1$ times to the left), the head is brought to the second symbol in the first half of the input. The scanned symbol is saved.
		\item Steps \ref{itm:double-step3} and \ref{itm:double-step4} are repeated until, after the single move to the right in step \ref{itm:double-step4}, the head scans the marked cell. When this happens, the comparison of the two halves has been completed.
		\item The machine performs $n$ steps to the right and, provided the scanned cell is $\rem$, accepts by moving to the right once more.
		      Otherwise, the input string is longer than $2n$, therefore the machine must reject.
	\end{enumerate}
	The resulting automaton, which only uses a counter up to $n$ and saves a symbol in $\set{a,b}$, has a linear number of states in $n$.

	We now prove that every \ONFA needs an exponential number of states to accept $\double$.
	In order to do so, we use a standard technique known as \emph{fooling set} \cite{Bir92}.
	Given a language $L$, a fooling set is a set of pairs of strings $\set{(x_i,y_i)}$ such that
	\begin{itemize}
		\item for all $i$, $x_iy_i\in L$, and
		\item for all $i\ne j$, $x_iy_j\notin L$ and $x_jy_i\notin L$.
	\end{itemize}
	The cardinality of each fooling set is a lower bound for the number of states of every \ONFA accepting $L$.

	Consider the set $S:=\set{(w_i,w_i)\mid w_i\in\set{a,b}^n}$.
	Obviously, $w_iw_i\in\double$ for all $i$.
	Furthermore, if $i\ne j$, $w_iw_j$ and $w_jw_i$ do not belong to $\double$, as the two halves of each word are different.
	Since $\card S=2^n$, every \ONFA accepting $\double$ has at least $2^n$ states.

	Finally, we point out that, by replacing the marked cell with a simple check that the input length is $2n$, the language $\double$ can be accepted by a simple \TDFA with a slightly greater number of states.
	Indeed, since \TDFA can be considered a special case of \OMODLA, the exponential cost of the simulation of \OMODLA by \ONFA can be simply considered a consequence of the one of \TDFA by \ONFA.
\end{proof}



\section{The unary case}\label{sec:oncemarking-unary}
The current known bounds for the simulations costs between unary once-marking $1$-limited automata and classical regular language acceptors are summarized in \Cref{tab:sims-om-unary-oncemarking}.

\begin{table}
	\makebox[\textwidth]{%
	\begin{minipage}{\dimexpr\textwidth+\oddsidemargin+\evensidemargin\relax}
		\centering
		\renewcommand{\arraystretch}{1.2}
		\renewcommand{\hstdef}{1.75}
		\begin{tabular}{|l|p{5.4em}|p{5.4em}|p{4.3em}|p{2.9em}|l|p{4.3em}|}
			\hline
			~       & \ODFA                              & \ONFA                             & \TDFA                                            & \TNFA                           & \OMOLA    & \OMODLA                                          \\ \hline
			\ODFA   & \cY                                & $\Cctriv$                         & $\Cctriv$                                        & $\Cctriv$                       & $\Cctriv$ & $\Cctriv$                                        \\ \hline
			\ONFA   & \rbt{$\CsubEq$}                    & \cY                               & \cR $\Theta(n^2)$                                & $\Cctriv$                       & $\Cctriv$ & \cB $O(n^2)$                                     \\ \hline
			\TDFA   & \rbt{$\CsubEq$}                    & \rbt{$\CsubEq$}                   & \cY                                              & $\Cctriv$                       & $\Cctriv$ & $\Cctriv$                                        \\ \hline
			\TNFA   & \rbt{$\CsubEq$}                    & \rbt{$\CsubEq$}                   & \cR \rbt[.4]{$\le\Csubln$} \rbt[.3]{$\ge\Cpoly$} & \cY                             & $\Cctriv$ & \cB \rbt[.4]{$\le\Csubln$} \rbt[.3]{$\ge\Cpoly$} \\ \hline
			\OMOLA  & $\le\Cdexp\hst[2.35]$ $\ge\CsubGe$ & $\le\Cexp\hst[2.85]$ $\ge\CsubGe$ & \cG $\le\Cexp\hst$ $\ge\Cpoly$                   & $\le\Cexp\hst[.35]$ $\ge\Cpoly$ & \cY       & \cG $\le\Cexp\hst$ $\ge\Cpoly$                   \\ \hline
			\OMODLA & $\le\Cexp\hst[2.85]$ $\ge\CsubGe$  & $\le\Cexp\hst[2.85]$ $\ge\CsubGe$ & $O(n^3)$                                         & $O(n^3)$                        & $\Cctriv$ & \cY                                              \\ \hline
		\end{tabular}
	\end{minipage}%
}
%
	\caption{Costs of the simulations between unary once-marking $1$-limited automata and other regular language acceptors.}
	\label{tab:sims-om-unary-oncemarking}
\end{table}

The bounds derive from results on either unary $1$-limited automata, general-case once-marking $1$-limited automata, or classical unary language acceptors.
Because among the latter some subexponential bounds appear, some of the non-tight bounds have a less than exponential distance (subexponential to exponential or polynomial to subexponential).
An explanation of each bound is given in \Cref{app:results}.

Working with subexponential bounds can be challenging, since, unlike with exponentials and polynomials, an operation between a subexponential and another function (product, composition, etc.) can produce another subexponential function or an exponential one, depending on the specific functions.
For example, by combining the exponential construction to simulate a \OLA with a \ONFA with the subexponential construction to simulate a \ONFA with a \ODFA, one might expect to obtain a cost that is less than double exponential, but because of the specific functions this is not the case.
The simulation of \OMOLA by \ODFA is particularly interesting as, as it was the case for general $1$-limited automata, many results can be obtained as consequences of either an improved simulation or a witness language that implies a tighter lower bound.
We investigated both these paths, and we conjecture that, in the unary case, an improved upper bound can be found for such simulation.



\section{Improving known bounds}\label{sec:oncemarking-ideas}
In order to research the possibility of improving the known bounds on the simulation of once-marking $1$-limited automata by one-way deterministic finite state automata, we considered techniques from both the unary and the regular language world.
As said above, we investigated the possible improvements of both the lower and the upper bounds.


\subsection{Lower bound}

\paragraph{Bounded regular languages} A first idea was the possibility of applying to the once-marking model techniques known for bounded regular languages.
A language is called (strongly) bounded if it is a subset of $\sigma_1\star\sigma_2\star\cdots\sigma_k\star$, where $\sigma_1,\sigma_2,\dots,\sigma_k$ are (pairwise distinct) symbols \cite{GinSpa66}.\footnote{%
	Results on the descriptional complexity of bounded languages can be found in \cite{Gin66,MalPig13,IbaRav16,HerKut+17}.}

Consider a unary once-marking $1$-limited automaton that receives an input $w\in\sigma\star$.
Once the automaton marks a cell, the string on the tape is divided in two parts $\sigma\cdots\sigma \tilde\sigma \sigma\cdots\sigma$, making it essentially equivalent to a string belonging to $\sigma_1\star\sigma_2\star$.
Of course, after it performs the write operation, the automaton is virtually a \TNFA, therefore only regular languages of this kind can be accepted.
Hence, bounded regular languages, specifically when $k=2$, are of interest for the analysis of unary \OMOLA.

Unfortunately, the techniques we know to produce lower bounds for bounded regular languages typically define $k$ in function of the parameter under consideration (number of states), while we would need a constant $k=2$.
An example is the "tool for constructing lower bound witnesses" proposed by \citeauthor{HerKut+17} in \cite{HerKut+17}.

\paragraph{Unary concatenation} An interesting solution to the problem of limiting $k$ to $2$ would be to obtain a witness language as a concatenation of two unary languages over different unary alphabets.
Unfortunately, the results we could find in literature mainly study the concatenation of two unary languages over the same alphabet, as it was the case in \cite{YuZhu+94,PigSha02}.
Besides, such results, if extended to different alphabets, would only imply a quadratic lower bound, which is worse than the currently known subexponential.

\paragraph{Chrobak's witness language} A different approach could be to study a variation of the well known witness language used by \citeauthor{Chr86} to prove many of the lower bounds on basic unary regular language acceptors \cite{Chr86}.
In particular, the concatenation of two similar (but different) witness languages might force a \ODFA to require a subexponential state component for each part of the input word, while a once-marking automaton could simply nondeterministically guess and mark the first symbol of the second part and separately verify that each of the two factors belongs to its respective language.


\subsection{Upper bound}

\paragraph{Transition tables} The creation of an exponential simulation based on transition tables similar to the one of \TNFA by \ODFA seems out of the question, as different computation paths can replace the same input prefix on the tape with different strings, therefore forcing different transition tables, as it is the case for the simulation of \OLA by \ODFA.

\paragraph{Limiting the marking position} A promising idea is the one of restricting the positions where the automaton can mark cells, specifically by limiting the distance from the end-markers where the writing transition can be performed.
This concept is inspired by the simulation of unary \TNFA by \ODFA presented by \triplecite{MerPig01}.
Here, they limit the distance from end-markers where the \TNFA can perform a "U-turn" (that is, changing direction of movement).
Specifically, they prove that there are constants $\ell$ and $m_0$, respectively subexponential and quadratic in the number of states of the \TNFA, such that a U-turn is possible in a position $i\ge m_0$ if and only if it is possible in position $i+\ell$.

We believe that a similar argument can be made to limit the positions where, in a unary once-marking $1$-limited automaton, the write operation is performed, thus reducing the size of a simulating machine.
We point out that if a subexponential simulation of \OMOLA by \ODFA was to be found, all the costs on one-way finite automata would collapse to subexponential.
Ideally, a polynomial simulation of \OMOLA by \TDFA would solve the "hard" variations of the Sakoda and Sipser problem in this realm (green in \Cref{tab:sims-om-unary-oncemarking}), and may also bring the one-way open costs to subexponential, depending on the specific function.
