\chapter{Preliminaries}
% TODO: context, explanation. Possibly figures



\section{Introduction}
\dots



\section{General notation}
The language accepted by a machine $M$ is denoted by $\genlang(M)$.

\begin{defn}
	A machine with set of states $S$, working alphabet $\Gamma$ and transition function $\delta:S\times\Gamma\to\subsets{T}$ is said to be deterministic whenever $\card{\delta(s,\gamma)}\le1$, for each $s\in S,\gamma\in\Gamma$.

	In this case, $\delta$ can be interpreted as a partial function, with the notation $\delta(s,\gamma)=t$ used as a shorthand for $\delta(s,\gamma)=\set{t}$.
\end{defn}

With an abuse of notation, we use $\delta$ to indicate both a transition function and its reflexive, transitive closure, possibly using strings for the second argument.

Given a word $w=w_1 w_2\cdots w_n$, the reversal of $w$ is $w\rev:=w_n w_{n-1}\cdots w_1$.
Given a language $L$, the reversal of $L$ is $L\rev:=\set{w\rev \mid w\in L}$.

A machine working over a single letter alphabet is called unary, as its generated language.

Given a transition function $\delta$, a set of states $S$, and a symbol $\sigma$, we define
\begin{equation*}
	\delta(S,\sigma):=\bigcup_{q\in S} \delta(q,\sigma) \text.
\end{equation*}



\section{One-way computational models}
\begin{defn}[one-way finite automata]
	A \emph{(one-way) nondeterministic finite automaton} (\ONFA) is a quintuple $(Q,\Sigma,\delta,q_I,F)$ where:
	\begin{itemize}
		\item $Q$ is the finite and nonempty set of \emph{states},
		\item $\Sigma$ is the finite \emph{input alphabet},
		\item $\delta:Q\times\Sigma\to\subsets Q$ is the \emph{transition function},
		\item $q_I\in Q$ is the \emph{initial state}, and
		\item $F\subseteq Q$ is the set of \emph{accepting} (or \emph{final}) \emph{states}.
	\end{itemize}
	The input word $w$ is read from left to right, one symbol at a time.
	At every step, the machine reads an input symbol and changes its state, with the transition function $\delta$ representing the possible next states based on the current state and the symbol read.
	The \ONFA accepts an input string $w=w_1\cdots w_n$ if there is a sequence of states $q_0,q_1,\dots,q_n$ such that $q_0=q_I$, $q_n\in F$, and for each $i\in\set{0,1,\dots,n-1}$, $q_{i+1}\in\delta(q_i,w_i)$.

	The deterministic version of \ONFA is denoted \ODFA.
\end{defn}



\section{Two-way computational models}

\begin{defn}[two-way finite automata]
	A \emph{two-way nondeterministic finite automaton} (\TNFA) is a quintuple $(Q,\Sigma,\delta,q_I,F)$ where:
	\begin{itemize}
		\item $Q$, $\Sigma$, $q_I$ and $F$ are defined as for \ONFAs, and
		\item $\delta:Q\times(\Sigma\cup\set{\lem,\rem})\to\subsets{Q\times\set{\tl,\tr}}$ is the \emph{transition function}, where the two special symbols $\lem,\rem\notin\Sigma$ are called the left and the right end-markers, respectively.
	\end{itemize}
	At the beginning of the computation, the input word $w$ is stored onto a tape surrounded by the two end-markers, the left end-marker being at position zero and the right end-marker at position $\len w+1$.
	The tape has a head, which initially points at $\lem$.
	This is known as the initial configuration of the machine.
	In one move, the machine reads an input symbol, changes its state, and moves the tape head one position left or right, depending on whether $\delta$ returns $\tl$ or $\tr$, respectively.
	The succession of configurations forms what is called a computation path.
	Furthermore, the head cannot pass the end-markers, except at the end of computation, to accept the input.
	The machine accepts the input if there exists a computation path that starts in the initial configuration and ends in a final state $q_f\in F$ after moving right from $\rem$.

	The deterministic version of \TNFA is denoted \TDFA.
\end{defn}


\begin{defn}[limited automata]\label{def:kla}
	Given an integer $k\ge0$, a \emph{$k$-limited automaton} (\kLA) is a 6-tuple $(Q,\Sigma,\Gamma,\delta,q_I,F)$ where:
	\begin{itemize}
		\item $Q$, $\Sigma$, $q_I$ and $F$ are defined as for \TNFAs,
		\item $\Gamma$ is the finite \emph{working alphabet}, partitioned into $\Gamma_0\cup\Gamma_1\cup\cdots\cup\Gamma_k$, where $\Gamma_0=\Sigma$, and
		\item $\delta:Q\times\Gamma\to\subsets{Q\times(\Gamma\setminus\set{\lem,\rem})\times\set{\tl,\tr}}$ is the \emph{transition function}.
	\end{itemize}
	The initial configuration is the same as for \TNFAs.
	In one move the machine, beside changing its state and moving the tape head, changes the content of scanned cell according to the transition function.

	Furthermore, the transition function is subject to restrictions which, essentially, allow to modify the content of a cell only during the first $k$ visits.
	In order to implement these restrictions, $\delta$ is required to satisfy the following conditions. For each $(t,y,d)\in\delta(s,x)$, with $x\in\Gamma_i$:
	\begin{enumerate}[(1)]
		\item if $i=k$ then $x=y$,
		\item if $i<k$ and $d=\tr$ then $y\in\Gamma_j$, with $j=\min\set{\ceil{\frac{i}{2}}\cdot 2+1,k}$, and
		\item if $i<k$ and $d=\tl$ then $y\in\Gamma_j$, with $j=\min\set{\ceil{\frac{i+1}{2}}\cdot 2,k}$.
	\end{enumerate}

	\noindent Acceptance is defined as for \TNFAs.

	\noindent The deterministic version of \kLA is denoted \kDLA.
\end{defn}


\begin{defn}\label{def:sweeping}
	A two-way machine is said to be \emph{sweeping} if the direction of the head movement only changes on the end-markers.
	The computation is therefore divided into traversals of the entire tape, also called \emph{sweeps}.
\end{defn}
