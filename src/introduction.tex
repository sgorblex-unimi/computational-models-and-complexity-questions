\chapter{Introduction}
After millennia of human history, in which we repeated the same actions over and over, people started wondering if those actions could be done automatically.
Computer Science was born with the prospect of answering two questions: "what problems can be solved automatically?", and "how many resources would it cost?".

Theoretical computer science developed throughout the years several ways of describing automation, one of them being computational models.
Such concepts are studied in Formal Language Theory, and consist in mathematical objects (machines) that process strings of characters.
Each of these machines can recognize what strings belong to a certain set, called the accepted language.
Different types of machines can accept different classes of languages, bringing to the questions of their equivalence and their comparison in size.
Descriptional Complexity studies ways of describing languages, including computational models, in terms of how succinct they can be.

The innermost class of the classic hierarchy of languages (the Chomsky hierarchy~\cite{Cho56}) is the one of regular languages, famously recognized by finite state automata, which are devices whose memory consists only of a finite state control and, hence, has a fixed size.
An important open problem of Descriptional Complexity is the Sakoda and Sipser conjecture~\cite{SakSip78}, regarding the size trade-off between nondeterministic and deterministic regular language acceptors.
The most powerful computational model is the Turing Machine, composed by a tape with an infinite number of cells, a head which scans and possibly writes on one cell at a time, and a finite state control. The tape can be used to store information, hence, the memory is unbounded.
Many restrictions of the Turing Machine have been studied, by comparing their accepting power with the ones of other models.

Limited automata were introduced by~\citeauthor{Hib67} in~\citeyear{Hib67} as a restriction of nondeterministic Turing Machines~\cite{Hib67}.
% Given an integer~$k$, a $k$-limited automaton has a finite tape, initially containing the input string and delimited by special symbols called end-markers, and can only write on each tape cell on its first~$k$ visits.
Given an integer~$k$, a $k$-limited automaton has a finite tape, initially containing the input string, and can only write on each tape cell on its first~$k$ visits.
Hibbard showed that for each $k\ge2$, $k$-limited automata accept the class of context-free languages~\cite{Hib67}.
In~\citeyear{WagWec86},~\citeauthor{WagWec86} proved that when~$k=1$ such machines accept regular languages~\cite{WagWec86}.
Later,~\citeauthor{PigPis14} proved that their simulation by deterministic finite state automata can require a double exponential growth in size, making $1$-limited automata much more succinct~\cite{PigPis14}.

The discovery of a new succinct model for regular languages sparked a new interest for limited automata, leading to the study of several variations of the model, with the hope of a new perspective on tackling the Sakoda and Sipser problem.
In this work, we study two models in particular: \emph{sweeping} $k$-limited automata, and \emph{once-marking} $1$-limited automata.



\section{Structure of the thesis}
The work is divided as follows:
\begin{description}
	\item[\Cref{ch:context}] gives an overview of the motivation behind our work and the history of the literature around the studied problems.
	\item[\Cref{ch:preliminaries}] formally describes the main concepts of automata theory and descriptional complexity that are relevant to our work.
	\item[\Cref{ch:techniques}] shows a few techniques used in the simulation of two-way finite automata by one-way finite automata.
	\item[\Cref{ch:sweeping}] presents a construction for the simulation of deterministic sweeping $k$-limited automata by nondeterministic finite state automata, based on the techniques showed in the previous chapter. We discuss descriptional complexity aspects, showing that the simulation has an exponential cost in size, and present a matching weak lower bound.
	\item[\Cref{ch:nondetsweep}] explains how the simulation in the previous chapter can be extended to the nondeterministic case.
	\item[\Cref{ch:oncemarking}] summarizes our investigations about once-marking limited automata, in our attempts to improve upper and lower bounds for their simulation by other equivalent recognizers.
	\item[\Cref{app:results}] summarizes descriptional complexity results around the simulations between regular language acceptors, including $1$-limited automata and once-marking limited automata.
	\item[\Cref{app:constructions}] contains few constructions on finite state automata that let us make simplifications across the main chapters.
\end{description}
